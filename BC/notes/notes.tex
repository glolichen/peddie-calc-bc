% JUMP TO LINE 60, 73
\documentclass[preview, margin=0.6in]{standalone}
\usepackage[letterpaper,portrait,top=0.4in, left=0.6in, right=0.6in, bottom=1in]{geometry}

\usepackage{amsmath, amsfonts, amsthm, amssymb}
\usepackage{graphicx, float}
\usepackage{mathtools}
\usepackage{titlesec}
\usepackage{interval}
\usepackage{hyperref}
\usepackage{siunitx}
\usepackage{titling}
\usepackage{vwcol}
\usepackage{setspace}
\usepackage{empheq}
\usepackage{cancel}
\usepackage{esdiff}
\usepackage{multicol}
\usepackage{mdframed}
\usepackage{esdiff}
\usepackage{tikzsymbols}
\usepackage{titlesec}
\usepackage{multicol}
\usepackage{tikz}
\usepackage{varwidth}
\usepackage{pgfplots}
\pgfplotsset{compat=1.18}
\intervalconfig {
	soft open fences
}

\newcommand{\alignedintertext}[1]{%
  \noalign{%
    \vskip\belowdisplayshortskip
    \vtop{\hsize=\linewidth#1\par
    \expandafter}%
    \expandafter\prevdepth\the\prevdepth
  }%
}

\newtheorem{lemma}{Lemma}

\renewcommand{\qedsymbol}{\Smiley[1.3]}
\newcommand*{\problem}[1]{\section*{Problem #1}}
\newcommand*{\aps}{\section*{AP Corner}}
\newcommand*{\deriv}[1][x]{\ensuremath{\dfrac{\mathrm{d}}{\mathrm{d}#1}}}
\newcommand*{\floor}[1]{\ensuremath{\lfloor #1\rfloor}}
\newcommand*{\lheqzero}{\ensuremath{\underset{\text{L'H}}{\overset{\left[\frac00\right]}{=}}}}
\newcommand*{\lheqinfty}{\ensuremath{\underset{\text{L'H}}{\overset{\left[\frac{\infty}{\infty}\right]}{=}}}}

\DeclareMathOperator{\DNE}{DNE}
\DeclareMathOperator{\sgn}{sgn}

\DeclareMathOperator{\arccsc}{arccsc}
\DeclareMathOperator{\arcsec}{arcsec}
\DeclareMathOperator{\arccot}{arccot}

\setlength{\parindent}{0pt}

\titleformat{\section} {\normalfont\fontsize{16}{14}\bfseries}{\thesection}{1em}{}
\titleformat{\subsection} {\normalfont\fontsize{14}{14}\bfseries}{\thesubsection}{1em}{}


%opening
\title{\vspace*{-30pt}AP Calculus BC Notes}
\author{Jayden Li}
\date{Fall Term, 2024-25}

% \allowdisplaybreaks
\postdisplaypenalty=100000

\begin{document}
\setstretch{1.5}
\fontsize{14pt}{14pt}\selectfont
\setlength{\abovedisplayskip}{4pt}
\maketitle

\section{The Fundemental Theorem of Calculus}

\subsection{Part 1: Definite integral as a function}

Let $x\in[a,b]$ and $f$ is an integrable function. Then we define the antiderivative $F$:
\begin{equation*}
	F(x)=\int_{a}^{x}f(t)\,\mathrm{d}t
\end{equation*}

$F$ is a function as the bounds of integration is a variable. The fundamental theorem of calculus states that:
\begin{gather*}
	F'(x)=\deriv \int_{a}^{x}f(x)\,\mathrm{d}x=f(x)
\end{gather*}

If the bounds of integration is the function $g$, we use the chain rule:
\begin{equation*}
    \deriv \int_{a}^{g(x)}f(t)\,\mathrm{d}t=\deriv F(g(x))=F'(g(x))g'(x)=f(g(x))g'(x)
\end{equation*}

\subsection{Part 2: Definite integral as a number}
Suppose $F$ is the antiderivative of $f$ (then $F'(x)=f(x)$). Then:
\begin{equation*}
    \int_{a}^{b}f(x)\,\mathrm{d}x=F(b)-F(a)
\end{equation*}

In this statement, $f(x)$ is called the \textit{integrand}. To evaluate the statement, we first find the antiderivative $F$.

\subsection{Net Change Theorem}
We can rewrite the second part of the FTC as follows:
\begin{align*}
    \int_{a}^{b}F'(x)\,\mathrm{d}x=F(b)-F(a)
\end{align*}
This is the \textit{Net Change Theorem}.


\end{document}
