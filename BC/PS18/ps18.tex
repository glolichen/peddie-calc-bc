% JUMP TO LINE 60, 73
% \documentclass[preview, margin=0.6in]{standalone}
\documentclass{article}
\usepackage[letterpaper,portrait,top=0.4in, left=0.6in, right=0.6in, bottom=1in]{geometry}

\usepackage{amsmath, amsfonts, amsthm, amssymb}
\usepackage{graphicx, float}
\usepackage{mathtools}
\usepackage{titlesec}
\usepackage{interval}
\usepackage{hyperref}
\usepackage{siunitx}
\usepackage{titling}
\usepackage{vwcol}
\usepackage{setspace}
\usepackage{empheq}
\usepackage{cancel}
\usepackage{esdiff}
\usepackage{multicol}
\usepackage{mdframed}
\usepackage{esdiff}
\usepackage{tikzsymbols}
\usepackage{multicol}
\usepackage{tikz}
\usepackage{varwidth}
\usepackage{pgfplots}
\pgfplotsset{compat=1.18}
\intervalconfig {
	soft open fences
}

\newcommand{\alignedintertext}[1]{%
  \noalign{%
    \vskip\belowdisplayshortskip
    \vtop{\hsize=\linewidth#1\par
    \expandafter}%
    \expandafter\prevdepth\the\prevdepth
  }%
}

\newtheorem{lemma}{Lemma}

\renewcommand{\qedsymbol}{\Smiley[1.3]}
\newcommand*{\problem}[1]{\section*{Problem #1}}
\newcommand*{\aps}{\section*{AP Corner}}
\newcommand*{\deriv}[1][x]{\ensuremath{\dfrac{\mathrm{d}}{\mathrm{d}#1}}}
\newcommand*{\floor}[1]{\ensuremath{\lfloor #1\rfloor}}
\newcommand*{\lheqzero}{\ensuremath{\underset{\text{L'H}}{\overset{\left[\frac00\right]}{=}}}}
\newcommand*{\lheqinfty}{\ensuremath{\underset{\text{L'H}}{\overset{\left[\frac{\infty}{\infty}\right]}{=}}}}

\DeclareMathOperator{\DNE}{DNE}
\DeclareMathOperator{\sgn}{sgn}

\DeclareMathOperator{\arccsc}{arccsc}
\DeclareMathOperator{\arcsec}{arcsec}
\DeclareMathOperator{\arccot}{arccot}

\setlength{\parindent}{0pt}

%opening
\title{\vspace*{-30pt}Problem Set \#18}
\author{Jayden Li}
\date{\today}

% \allowdisplaybreaks
\postdisplaypenalty=100000

\begin{document}
\setstretch{1.25}
\fontsize{12pt}{12pt}\selectfont
\setlength{\abovedisplayskip}{0pt}
\maketitle
\problem{1}
\begin{itemize}
	\item[(b)]
		$\begin{aligned}[t]
		    \int \frac{6x-2}{x^2-2x-3}\,\mathrm{d}x
			&=\int \frac{6x-2}{(x-3)(x+1)}\,\mathrm{d}x
			=\int\left(\frac{A}{x-3}+\frac{B}{x+1}\right)\mathrm{d}x \\
			\alignedintertext{
				\begin{mdframed}
					$\begin{aligned}[t]
						\frac{A}{x-3}+\frac{B}{x+1}=\frac{6x-2}{x^2-2x-3}
						&\implies Ax+A+Bx-3B=6x-2
						\implies \left\{\begin{aligned}
							A+B&=6 \\
							A-3B&=-2
						\end{aligned}\right. \\
						&\implies 4B=8
						\implies B=2
						\implies A=4
					\end{aligned}$
				\end{mdframed}
			}
			&=\int\left(\frac{4}{x-3}+\frac{2}{x+1}\right)\mathrm{d}x
			=\boxed{4\ln|x-3|+2\ln|x+1|+C}
		\end{aligned}$

	\item[(c)]
		$\begin{aligned}[t]
			\int \frac{2x+5}{x^2+2x-8}\,\mathrm{d}x
			&=\int \frac{2x+5}{(x+4)(x-2)}\,\mathrm{d}x
			=\int\left(\frac{A}{x+4}+\frac{B}{x-2}\right)\mathrm{d}x \\
			\alignedintertext{
				\begin{mdframed}
					$\begin{aligned}[t]
						\frac{A}{x+4}+\frac{B}{x-2}=\frac{2x+5}{x^2+2x-8}
						&\implies Ax-2A+Bx+4B=2x+5
						\implies \left\{\begin{aligned}
							A+B&=2 \\
							-2A+4B&=5
						\end{aligned}\right. \\
						&\implies 6B=9
						\implies B=\frac{3}{2}
						\implies A=\frac{1}{2}
					\end{aligned}$
				\end{mdframed}
			}
			&=\int\left(\frac{1/2}{x+4}+\frac{3/2}{x-2}\right)\mathrm{d}x
			=\boxed{\frac12 \ln|x+4|+\frac32\ln|x-2|+C}
		\end{aligned}$
\end{itemize}

\problem{2}
\begin{itemize}
	\item[(b)]
		$\begin{aligned}[t]
			\int \frac{x^2+4x-1}{x^3-x}\,\mathrm{d}x
			&=\int \frac{x^2+4x-1}{x(x+1)(x-1)}\,\mathrm{d}x
			=\int\left(\frac{A}{x}+\frac{B}{x+1}+\frac{C}{x-1}\right)\mathrm{d}x \\
			\alignedintertext{
				\begin{mdframed}
					$\begin{aligned}[t]
						&\frac Ax+\frac{B}{x+1}+\frac{C}{x-1}=\frac{x^2+4x-1}{x^3-x} \\
						\implies{}& A(x+1)(x-1)+Bx(x-1)+Cx(x+1)=x^2+4x-1 \\
						\implies{}& Ax^2-A+Bx^2-Bx+Cx^2+Cx=x^2+4x-1
						\implies \left\{\begin{aligned}
								A+B+C&=1 \\
								-B+C&=4 \\
								-A&=-1
						\end{aligned}\right. \\
						\implies{}& A=1
						\implies \left\{\begin{aligned}
								B+C&=0 \\
								-B+C&=4
						\end{aligned}\right.
						\implies 2C=4
						\implies C=2
						\implies B=-2
					\end{aligned}$
				\end{mdframed}
			}
			&=\int\left(\frac{1}{x}-\frac{2}{x+1}+\frac{2}{x-1}\right)\mathrm{d}x
			=\boxed{\ln|x|-2\ln|x+1|+2\ln|x-1|+C}
		\end{aligned}$

	\item[(c)]
		$\begin{aligned}[t]
			\int \frac{4x+28}{(x+1)(x^2-4x+3)}\,\mathrm{d}x
			&=\int \frac{4x+28}{(x+1)(x-1)(x-3)}\,\mathrm{d}x
			=\int\left(\frac{A}{x+1}+\frac{B}{x-1}+\frac{C}{x-3}\right)\mathrm{d}x \\
			\alignedintertext{
				\begin{mdframed}
					$\begin{aligned}[t]
						&\frac{A}{x+1}+\frac{B}{x-1}+\frac{C}{x-3}=\frac{4x+28}{(x+1)(x-1)(x-3)} \\
						\implies{}& A(x-1)(x-3)+B(x+1)(x-3)+C(x+1)(x-1)=4x+28 \\
						\implies{}& A \left(x^2-4x+3\right)+B \left(x^2-2x-3\right)+C \left(x^2-1\right)=4x+28 \\
						\implies{}& Ax^2-4Ax+3A+Bx^2-2Bx-3B+Cx^2-C=4x+28
						\implies\left\{\begin{aligned}
								A+B+C&=0 \\
								-4A-2B&=4 \\
								3A-3B-C&=28
						\end{aligned}\right. \\
						\implies{}&\left\{\begin{aligned}
								4A-2B&=28 \\
								-4A-2B&=4
						\end{aligned}\right.
						\implies -4B=32
						\implies B=-8
						\implies A=3
						\implies C=5
					\end{aligned}$
				\end{mdframed}
			}
			&=\int\left(\frac{3}{x+1}-\frac{8}{x-1}+\frac{5}{x-3}\right)\mathrm{d}x \\
			&=\boxed{3\ln|x+1|-8\ln|x-1|+5\ln|x-3|+C}
		\end{aligned}$
\end{itemize}

\problem{3}
\begin{itemize}
	\item[(a)]
		$\begin{aligned}[t]
			\int \frac{3x^2-7x+2}{x^3-2x^2+x}\,\mathrm{d}x
			&=\int \frac{3x^2-7x+2}{x(x-1)(x-1)}\,\mathrm{d}x
			=\int \left(\frac{A}{x}+\frac{B}{x-1}+\frac{C}{(x-1)^2}\right)\,\mathrm{d}x \\
			\alignedintertext{
				\begin{mdframed}
					$\begin{aligned}[t]
						&\frac{A}{x}+\frac{B}{x-1}+\frac{C}{(x-1)^2}=\frac{3x^2-7x+2}{x(x-1)(x-1)} \\
						\implies{}& \frac{A(x-1)(x^2-2x+1)+Bx(x^2-2x+1)+Cx(x-1)}{x(x-1)(x-1)^2}=\frac{3x^2-7x+2}{x(x-1)(x-1)} \\
						\implies{}& A \left(x^3-3x^2+3x-1\right)+B \left(x^3-2x^2+x\right)+C \left(x^2-x\right)=(x-1)\left(3x^2-7x+2\right) \\
						\implies{}& Ax^3-3Ax^2+3Ax-A+Bx^3-2Bx^2+Bx+Cx^2-Cx=3x^3-10x^2+9x-2 \\
						\implies{}& \left\{\begin{aligned}
								A+B&=3 \\
								-3A-2B+C&=-10 \\
								3A+B-C&=9 \\
								-A&=-2
						\end{aligned}\right.
						\implies A=2
						\implies B=1
						\implies C=-2
					\end{aligned}$
				\end{mdframed}
			}
			&=\int \left(\frac{2}{x}+\frac{1}{x-1}-\frac{2}{(x-1)^2}\right)\,\mathrm{d}x
			=\boxed{2\ln|x|+\ln|x-1|+\frac{2}{x-1}+C}
		\end{aligned}$

	\item[(b)]
		$\begin{aligned}[t]
		    \int \frac{3x^2-2x-3}{x^3-x^2}\,\mathrm{d}x
			&=\int \frac{3x^2-2x-3}{x^2(x-1)}\,\mathrm{d}x
			=\int\left(\frac{A}{x}+\frac{B}{x^2}+\frac{C}{x-1}\right)\mathrm{d}x \\
			\alignedintertext{
				\begin{mdframed}
					$\begin{aligned}[t]
						&\frac{A}{x}+\frac{B}{x^2}+\frac{C}{x-1}=\frac{3x^2-2x-3}{x^2(x-1)}
						\implies \frac{Ax^2(x-1)+Bx(x-1)+Cx^3}{x^3(x-1)}=\frac{3x^2-2x-3}{x^2(x-1)} \\
						\implies{}& Ax^3-Ax^2+Bx^2-Bx+Cx^3=3x^3-2x^2-3x
						\implies\left\{\begin{aligned}
								A+C&=3 \\
								-A+B&=-2 \\
								-B&=-3
						\end{aligned}\right. \\
						\implies{}& B=3
						\implies A=5
						\implies C=-2
					\end{aligned}$
				\end{mdframed}
			}
			&=\int\left(\frac{5}{x}+\frac{3}{x^2}-\frac{2}{x-1}\right)\mathrm{d}x
			=\boxed{5\ln|x|-\frac 3x-2\ln|x-1|+C}
		\end{aligned}$

	\item[(c)]
		$\begin{aligned}[t]
		    \int \frac{x^2}{(x+1)^3}\,\mathrm{d}x
			&\left[\begin{aligned}
					t&=x+1 \\
					\mathrm{d}t&=\mathrm{d}x
			\end{aligned}\right]
			\int \frac{(t-1)^2}{t^3}\,\mathrm{d}x
			=\int \frac{t^2-2t+1}{t^3}\,\mathrm{d}t
			=\int \left(\frac{1}{t}-\frac{2}{t^2}+\frac{1}{t^3}\right)\,\mathrm{d}t \\
			&=\ln|t|+\frac2t+\frac{-1/2}{t^2}
			=\boxed{\ln|x+1|+\frac{2}{x+1}-\frac{1}{2(x+1)^2}+C}
		\end{aligned}$
\end{itemize}

\problem{4}
$\begin{aligned}[t]
    \int \ln\left(x^2-x+2\right)\mathrm{d}x
	&=\left[\begin{alignedat}{2}
			u&=\ln\left(x^2-x+2\right) &\quad \mathrm{d}u&=\frac{(2x-1)\,\mathrm{d}x}{x^2-x+2} \\
			\mathrm{d}v&=\mathrm{d}x & v&=x
	\end{alignedat}\right]
	x \ln\left(x^2-x+2\right)-\int \frac{2x^2-x}{x^2-x+2}\,\mathrm{d}x \\
	&=x \ln\left(x^2-x+2\right)-\int \left(2+\frac{x-4}{x^2-x+2}\right)\mathrm{d}x \\
	&=x \ln\left(x^2-x+2\right)-2x-\frac12\int \left(\frac{2x-1}{x^2-x+2}-\frac{7}{x^2-x+2}\right)\mathrm{d}x \\
	&=\left[\begin{aligned}
			t&=x^2-x+2 \\
			\mathrm{d}t&=(2x-1)\,\mathrm{d}x
	\end{aligned}\right]
	x \ln\left(x^2-x+2\right)-2x-\frac12 \int \frac{1}{t}\,\mathrm{d}t+\frac12 \int \frac{7}{x^2-x+2}\,\mathrm{d}x \\
	&=x \ln\left(x^2-x+2\right)-2x-\frac12 \ln \left|x^2-x+2\right|+\frac12 \int \frac{7}{x^2-x+2}\,\mathrm{d}x \\
	\alignedintertext{
		\begin{mdframed}
			$\begin{aligned}[t]
			    \int \frac{7}{x^2-x+2}\,\mathrm{d}x
				&=\int \frac{7}{\left(x-\frac12\right)^2-\frac14+2}\,\mathrm{d}x
				=\left[\begin{aligned}
						t&=x-\frac12 \\
						\mathrm{d}t&=\mathrm{d}x
				\end{aligned}\right]
				7\int \frac{1}{t^2+\frac74}\,\mathrm{d}t
				=7 \int \frac{1}{\frac74 \left(\frac47t^2+1\right)}\,\mathrm{d}t \\
				&=7\cdot\frac47 \int \frac{1}{\frac47t^2+1}\,\mathrm{d}t
				=\left[\begin{aligned}
						s&=\frac{2}{\sqrt{7}}t \\
						\mathrm{d}s&=\frac{2}{\sqrt{7}}\,\mathrm{d}t
				\end{aligned}\right]
				4\cdot \frac{\sqrt{7}}{2}\int \frac{1}{s^2+1}\,\mathrm{d}s
				=2\sqrt7\arctan s \\
				&=2\sqrt7 \arctan \left(\frac{2}{\sqrt{7}}t\right)
				=2\sqrt7 \arctan \left(\frac{2}{\sqrt{7}}\left(x-\frac12\right)\right)+C
			\end{aligned}$	
		\end{mdframed}
	}
	&=x \ln\left(x^2-x+2\right)-2x-\frac12 \ln \left|x^2-x+2\right|+\frac12 \left(2\sqrt7 \arctan \left(\frac{2}{\sqrt{7}}\left(x-\frac12\right)\right)\right) \\
	&=\boxed{x \ln\left(x^2-x+2\right)-2x-\frac12 \ln \left|x^2-x+2\right|+\sqrt7 \arctan \left(\frac{2}{\sqrt{7}}\left(x-\frac12\right)\right)+C}
\end{aligned}$

\problem{5}
\begin{mdframed}
	\begin{minipage}[t]{0.33\linewidth} 
		\begin{align*}
			\cos\left(2\cdot \frac{\theta}{2}\right)&=2\cos^2 \left(\frac{\theta}{2}\right)-1 \\
			\cos\theta+1&=2\cos^2 \left(\frac{\theta}{2}\right) \\
			\cos \frac{\theta}{2}&=\pm\sqrt{\frac{\cos\theta+1}{2}}
			\intertext{$\cos(\theta/2)\geq0$ if $\theta\in[-\pi,\pi]$}
			\cos \frac{\theta}{2}&=\boxed{\sqrt{\frac{\cos\theta+1}{2}}}
		\end{align*}
	\end{minipage}
	\begin{minipage}[t]{0.33\linewidth} 
		\begin{align*}
			\cos\left(2\cdot \frac{\theta}{2}\right)&=1-2\sin^2 \left(\frac{\theta}{2}\right) \\
			2\sin^2 \left(\frac{\theta}{2}\right)&=1-\cos\theta \\
			\sin \frac{\theta}{2}&=\boxed{\pm \sqrt{\frac{1-\cos\theta}{2}}}
		\end{align*}
	\end{minipage}
	\begin{minipage}[t]{0.33\linewidth} 
		\begin{align*}
			\tan \frac{\theta}{2}&=\sqrt{\frac{1-\cos\theta}{2}}/ \sqrt{\frac{1+\cos\theta}{2}} \\
								 &=\frac{\sqrt{1-\cos\theta}\sqrt{1-\cos\theta}}{\sqrt{1+\cos\theta}\sqrt{1-\cos\theta}} \\
								 &=\frac{1-\cos\theta}{\sqrt{1-\cos^2\theta}}
								 =\boxed{\frac{1-\cos\theta}{\sin\theta}}
		\end{align*}
	\end{minipage}
\end{mdframed}

\begin{itemize}
	\item[(a)]
		$\begin{aligned}[t]
		    \frac{1}{\sqrt{1+t^2}}
			&=\left(\sqrt{1+\left(\frac{1-\cos\theta}{\sin\theta}\right)^2}\right)^{-1}
			=\left(\sqrt{\frac{\sin^2\theta}{\sin^2\theta}+\frac{1-2\cos\theta+\cos^2\theta}{\sin^2\theta}}\right)^{-1} \\
			&=\sqrt{\left(\frac{1-2\cos\theta+\cos^2\theta+\sin^2\theta}{\sin^2\theta}\right)^{-1}}
			=\sqrt{\frac{1-\cos^2\theta}{2-2\cos\theta}}
			=\sqrt{\frac12\cdot \frac{(1+\cos\theta)(1-\cos\theta)}{1-\cos\theta}} \\
			&=\sqrt{\frac{1+\cos\theta}{2}}
			=\cos \frac{\theta}{2} \\
		    \frac{t}{\sqrt{1+t^2}}
			&=\tan \frac{\theta}{2}\cdot \sqrt{\frac{1+\cos\theta}{2}}
			=\pm\sqrt{\frac{(1-\cos\theta)^2(1+\cos\theta)}{2\sin^2\theta}}
			=\pm\sqrt{\frac{(1-\cos\theta)^2(1+\cos\theta)}{2(1+\cos\theta)(1-\cos\theta)}} \\
			&=\pm\sqrt{\frac{1-\cos\theta}{2}}
			=\sin \frac{\theta}{2}
		\end{aligned}$

	\item[(b)]
		$\begin{aligned}[t]
		    \frac{1-t^2}{1+t^2}
			&=\frac{\displaystyle\frac{\sin^2 x}{\sin^2 x}-\frac{(1-\cos  x)^2}{\sin^2 x}}{\displaystyle\frac{\sin^2 x}{\sin^2 x}+\frac{(1-\cos x)^2}{\sin^2 x}}
			=\frac{\sin^2 x-\left(1-2\cos x+\cos^2 x\right)}{\sin^2 x+\left(1-2\cos x+\cos^2 x\right)}
			=\frac{\sin^2 x-1+2\cos x-\cos^2 x}{\sin^2 x+1-2\cos x+\cos^2 x} \\
			&=\frac{1-\cos^2 x-1+2\cos x-\cos^2 x}{2-2\cos x} 
			=\frac{2\cos x-2\cos^2 x}{2-2\cos x} 
			=\frac{\cos( x)(1-\cos x)}{1-\cos x}
			=\cos x \\
			\frac{2t}{1+t^2}
			&=\frac{\displaystyle \frac{2-2\cos x}{\sin x}}{\displaystyle \frac{\sin^2 x}{\sin^2 x}+\frac{(1-\cos x)^2}{\sin^2 x}}
			=\frac{2-2\cos x}{\sin^2x+1-2\cos x+\cos^2x}\cdot\sin x
			=\frac{2-2\cos x}{2-2\cos x}\cdot\sin x
			=\sin x
		\end{aligned}$

	\item[(c)]
		$\begin{aligned}
			t&=\tan \left(\frac x2\right)
			\implies\arctan t=\frac x2
			\implies\frac{1}{1+t^2}\,\mathrm{d}t=\frac12 \,\mathrm{d}x
			\implies\mathrm{d}x=\frac{2}{1+t^2}\,\mathrm{d}t
		\end{aligned}$
\end{itemize}

\problem{6}
$\begin{aligned}[t]
    \int_{1}^{2}\frac{x^2+1}{3x-x^2}\,\mathrm{d}x
	&=\int_{1}^{2}\left(-1+\frac{3x+1}{x(3-x)}\right)\mathrm{d}x
	=\int_{1}^{2}\left(-1\right)\mathrm{d}x+\int_{1}^{2}\left(\frac{A}{x}+\frac{B}{3-x}\right)\mathrm{d}x \\
	\alignedintertext{
		\begin{mdframed}
			$\begin{aligned}[t]
				\frac{A}{x}+\frac{B}{3-x}=\frac{3x+1}{x(3-x)}
				&\implies 3A-Ax+Bx=3x+1
				\implies \left\{\begin{aligned}
						-A+B&=3 \\
						3A&=1
				\end{aligned}\right.
				\implies A=\frac13 \\
				&\implies -\frac13+B=3
				\implies B=\frac{10}{3}
			\end{aligned}$
		\end{mdframed}
	}
	&=\left[-x\right]_{1}^{2}+\int_{1}^{2}\left(\frac{1/3}{x}+\frac{10/3}{3-x}\right)\mathrm{d}x
	=-1+\left[\frac13\ln|x|-\frac{10}{3}\ln|3-x|\right]_{1}^{2} \\
	&=-1+\frac13\ln2-\cancel{\frac{10}{3}\ln1}-\cancel{\frac{1}{3}\ln1}+\frac{10}{3}\ln2
	=\boxed{\frac{11}{3}\ln2-1}
\end{aligned}$

\end{document}
