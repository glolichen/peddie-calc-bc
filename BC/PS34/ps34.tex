% JUMP TO LINE 60, 73
\documentclass[preview, margin=0.6in]{standalone}
\usepackage[letterpaper,portrait,top=0.4in, left=0.6in, right=0.6in, bottom=1in]{geometry}

\usepackage{amsmath, amsfonts, amsthm, amssymb}
\usepackage{graphicx, float}
\usepackage{mathtools}
\usepackage{titlesec}
\usepackage{interval}
\usepackage{hyperref}
\usepackage{siunitx}
\usepackage{titling}
\usepackage{vwcol}
\usepackage{setspace}
\usepackage{empheq}
\usepackage{cancel}
\usepackage{esdiff}
\usepackage{multicol}
\usepackage{mdframed}
\usepackage{esdiff}
\usepackage{tikzsymbols}
\usepackage{multicol}
\usepackage{tikz}
\usepackage{varwidth}
\usepackage{pgfplots}
\usepackage{parskip}
\pgfplotsset{compat=1.18}
\intervalconfig {
	soft open fences
}

\newcommand{\alignedintertext}[1]{%
  \noalign{%
    \vskip\belowdisplayshortskip
    \vtop{\hsize=\linewidth#1\par
    \expandafter}%
    \expandafter\prevdepth\the\prevdepth
  }%
}

\newtheorem{lemma}{Lemma}

\renewcommand{\qedsymbol}{\Smiley[1.3]}
\newcommand*{\problem}[1]{\section*{Problem #1}}
\newcommand*{\aps}{\section*{AP Corner}}
\newcommand*{\deriv}[1][x]{\ensuremath{\dfrac{\mathrm{d}}{\mathrm{d}#1}}}
\newcommand*{\floor}[1]{\ensuremath{\lfloor #1\rfloor}}
\newcommand*{\lheqzero}{\ensuremath{\underset{\text{L'H}}{\overset{\left[\frac00\right]}{=}}}}
\newcommand*{\lheqinfty}{\ensuremath{\underset{\text{L'H}}{\overset{\left[\frac{\infty}{\infty}\right]}{=}}}}

\DeclareMathOperator{\DNE}{DNE}
\DeclareMathOperator{\sgn}{sgn}

\DeclareMathOperator{\arctanh}{arctanh}
\DeclareMathOperator{\arccsc}{arccsc}
\DeclareMathOperator{\arcsec}{arcsec}
\DeclareMathOperator{\arccot}{arccot}

\setlength{\parindent}{0pt}

%opening
\title{\vspace*{-30pt}Problem Set \#34}
\author{Jayden Li}
\date{\today}

% \allowdisplaybreaks
\postdisplaypenalty=100000

\begin{document}
\setstretch{1.25}
\fontsize{12pt}{12pt}\selectfont
\setlength{\abovedisplayskip}{0pt}
\maketitle

\problem{4}
Let $M(t)$ be the mass of dissolved salt in kg. $\mathrm{d}M/\mathrm{d}t$ in kg/min. Water exits the tank at $10\si{\liter}/\si{\min}$.

$\begin{aligned}[t]
	&\diff Mt=-\frac{M(t)\si{\kilo\gram}}{1000\si{\liter}}\cdot \frac{10\si{\liter}}{\min}=-\frac{M(t)}{100}
	\implies \frac{1}{M(t)}\,\mathrm{d}M=-\frac{1}{100}\,\mathrm{d}t
	\implies \int\frac{1}{M(t)}\,\mathrm{d}M=-\int\frac{1}{100}\,\mathrm{d}t \\
	\implies{}&\ln|M(t)|=-\frac{t}{100}+C
	\implies M(t)=Ce^{-t/100} \\
	\implies{}&M(0)=15
	\implies 15=Ce^0
	\implies C=15
\end{aligned}$

\begin{itemize}
	\item[(a)] $\displaystyle M(t)=15e^{-t/100}$
	\item[(b)] $\displaystyle M(20)=15e^{-1/5}=12.281\si{kg}$
\end{itemize}

\problem{5}
Let $A(t)$ be the amount of alcohol in the vat. Initial concentration is $500\cdot 4\%=500\cdot0.04=20$

$\begin{aligned}[t]
	&\diff At=5\cdot0.06-\frac{A(t)}{500}\cdot5=0.3-0.01A(t)
	\implies \int \frac{1}{0.3-0.01A(t)}\,\mathrm{d}A=\int \mathrm{d}t \\
	\implies{}& -100\ln|0.3-0.01A(t)|=t+C
	\implies \left|0.3-\frac{1}{100}A(t)\right|=\exp \left(-\frac{t}{100}+C\right) \\
	\implies{}& 0.3-\frac{1}{100}A(t)=Ce^{-t/100}
	\implies A(t)=30-Ce^{-t/100} \\
	\implies{}& A(0)=30-Ce^{-0/100}=20
	\implies 30-20=C
	\implies C=10 \\
	\implies{}&\frac{A(60)}{500}=\frac{30-10e^{-60/100}}{500}
	=\frac{30-10e^{-3/5}}{500}
	\approx 0.049
	=\boxed{4.9\%}
\end{aligned}$

\problem{6}
We have that $\mathrm{d}m/\mathrm{d}t=km(t)$. Obviously the raindrop starts with a velocity of zero.

$\begin{aligned}[t]
	& \deriv[t]mv
	=m\diff vt+v\diff mt=\cancel{m(t)}\diff vt+vk\cancel{m(t)}=g\cancel{m(t)}
	\implies \diff vt=g-kv
	\implies \frac{1}{g-kv}\diff vt=1 \\
	\implies{}& \int \frac{1}{g-kv}\,\mathrm{d}v=\int\mathrm{d}t
	\implies -\frac1k\ln|g-kv|=t+C
	\implies \exp(\ln|g-kv|)=\exp(-kt+C) \\
	\implies{}& |g-kv|=e^Ce^{-kt}
	\implies g-kv=Ce^{-kt}
	\implies v(t)=\frac{g-Ce^{-kt}}k \\
	&v(0)=g-Ce^{0}=0
	\implies g-C=0
	\implies C=g \\
	\implies{}& \lim_{t\to \infty}v(t)=\lim_{t\to \infty}\frac{g-\cancel{ge^{-kt}}}k=\boxed{\frac gk}
\end{aligned}$

\problem{7}
\begin{itemize}
	\item[(a)] Obviously:
		\begin{align*}
			\diff At&=k \sqrt{A(t)}(M-A(t)) \\
			\diff[2]At&=k\cdot A'(t) \frac{1}{2 \sqrt{A(t)}} (M-A(t))+k \sqrt{A(t)}(-A'(t)) \\
					  &=\frac{kA'(t)(M-A(t))}{2 \sqrt{A(t)}}-kA'(t) \sqrt{A(t)}
		\end{align*}
		When $A(t)=\dfrac M3$, then $\displaystyle A'(t)=k \sqrt{\frac M3} \left(M-\frac M3\right)=k \frac{2M}{3}\sqrt{\frac M3}$. Then:

		\begin{align*}
		    A''\left(\frac M3\right)
			&=k \left(\frac{k \frac{2M}{3}\sqrt{\frac M3} \left(M-\frac M3\right)}{2 \sqrt{\frac M3}}-k \frac{2M}{3}\sqrt{\frac M3} \sqrt{\frac{M}{3}}\right) \\
			&=k^2 \left(\frac{\frac{2M}{3}\cdot \frac{2M}{3}}{2}-\frac{2M}{3}\cdot \frac M3\right)
			=k^2 \left(\frac{2M}{9}-\frac{2M}{9}\right)
			=0
		\end{align*}

		So $A(t)=M/3$ is a critical value and probably a local maximum of $A''$, but I am too lazy to do the second (third) derivative test.

	\item[(b)]
		$\begin{aligned}[t]
			&\diff At=k \sqrt{A(t)}(M-A(t))
			\implies \int\frac{1}{\sqrt{A(t)}(M-A(t))}\,\mathrm{d}A=\int k \,\mathrm{d}t \\
			\implies{}& \frac{2\arctanh \left(\frac{\sqrt{A}}{\sqrt{M}}\right)}{\sqrt{M}}=kt+C
			\implies \arctanh \left(\frac{\sqrt{A}}{\sqrt{M}}\right)=\frac{\sqrt{M}}{2}kt+C \\
			\implies{}&\frac{\sqrt{A}}{\sqrt{M}}=\tanh \left(\frac{\sqrt{M}}{2}kt+C\right)
			\implies \frac{A}{M}=\tanh^2 \left(\frac{\sqrt{M}}{2}kt+C\right) \\
			\implies{}& \boxed{A=M\tanh^2 \left(\frac{\sqrt{M}}{2}kt+C\right)}
		\end{aligned}$
\end{itemize}

\end{document}
