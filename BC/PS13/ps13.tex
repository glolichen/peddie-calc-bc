% JUMP TO LINE 60, 73
\documentclass[preview, margin=0.6in]{standalone}
\usepackage[letterpaper,portrait,top=0.4in, left=0.6in, right=0.6in, bottom=1in]{geometry}

\usepackage{amsmath, amsfonts, amsthm, amssymb}
\usepackage{graphicx, float}
\usepackage{mathtools}
\usepackage{titlesec}
\usepackage{interval}
\usepackage{hyperref}
\usepackage{siunitx}
\usepackage{titling}
\usepackage{vwcol}
\usepackage{setspace}
\usepackage{empheq}
\usepackage{cancel}
\usepackage{esdiff}
\usepackage{multicol}
\usepackage{mdframed}
\usepackage{esdiff}
\usepackage{tikzsymbols}
\usepackage{multicol}
\usepackage{tikz}
\usepackage{varwidth}
\usepackage{pgfplots}
\pgfplotsset{compat=1.18}
\intervalconfig {
	soft open fences
}

\newcommand{\alignedintertext}[1]{%
  \noalign{%
    \vskip\belowdisplayshortskip
    \vtop{\hsize=\linewidth#1\par
    \expandafter}%
    \expandafter\prevdepth\the\prevdepth
  }%
}

\newtheorem{lemma}{Lemma}

\renewcommand{\qedsymbol}{\Smiley[1.3]}
\newcommand*{\problem}[1]{\section*{Problem #1}}
\newcommand*{\aps}{\section*{AP Corner}}
\newcommand*{\deriv}[1][x]{\ensuremath{\dfrac{\mathrm{d}}{\mathrm{d}#1}}}
\newcommand*{\floor}[1]{\ensuremath{\lfloor #1\rfloor}}
\newcommand*{\lheqzero}{\ensuremath{\underset{\text{L'H}}{\overset{\left[\frac00\right]}{=}}}}
\newcommand*{\lheqinfty}{\ensuremath{\underset{\text{L'H}}{\overset{\left[\frac{\infty}{\infty}\right]}{=}}}}

\DeclareMathOperator{\DNE}{DNE}
\DeclareMathOperator{\sgn}{sgn}

\DeclareMathOperator{\arccsc}{arccsc}
\DeclareMathOperator{\arcsec}{arcsec}
\DeclareMathOperator{\arccot}{arccot}

\setlength{\parindent}{0pt}

%opening
\title{\vspace*{-30pt}Problem Set \#13}
\author{Jayden Li}
\date{\today}

% \allowdisplaybreaks
\postdisplaypenalty=100000

\begin{document}
\setstretch{1.25}
\fontsize{12pt}{12pt}\selectfont
\setlength{\abovedisplayskip}{0pt}
\maketitle
\problem{1}
\begin{itemize}
	\item[(c)]
	\begin{align*}
		\int \frac{(\ln x)^2}{x}\,\mathrm{d}x
		=\left[\begin{aligned}
			u&=\ln x \\
			\mathrm du&=\frac{\mathrm dx}{x}
		\end{aligned}\right]
		\int u^2\,\mathrm{d}u
		=\frac{u^3}{3}
		=\boxed{\frac{\left(\ln x\right)^3}{3}+C}
	\end{align*}

	\item[(d)]
	\begin{align*}
	    \int \frac{\sin\left(\ln x\right)}{x}\,\mathrm{d}x
		=\left[\begin{aligned}
			u&=\ln x \\ 
			\mathrm du&=\frac{\mathrm dx}{x}
		\end{aligned}\right]
		\int \sin\left(u\right)\,\mathrm{d}u
		=-\cos\left(u\right)
		=\boxed{-\cos\left(\ln x\right)+C}
	\end{align*}
\end{itemize}

\problem{2}
\begin{itemize}
	\item[(b)]
	\begin{align*}
		\int \frac{x}{36+25x^4}\,\mathrm{d}x
		&=\frac{1}{36}\int \frac{x}{1+\frac{25}{36}x^4}\,\mathrm{d}x
		=\left[\begin{aligned}
			u&=\frac{5}{6}x^2 \\ 
			\mathrm{d}u&=\frac{5}{3}x\,\mathrm{d}x
		\end{aligned}\right]
		\frac{1}{36}\cdot \frac{3}{5}\int \frac{1}{1+u^2}\,\mathrm{d}u \\
		&=\frac{1}{60}\arctan\left(u\right)
		=\boxed{\frac{1}{60}\arctan\left(\frac{5}{6}x^2\right)+C}
	\end{align*}

	\item[(f)]
	\begin{align*}
	    \int \frac{\sec^2\left(4x\right)}{9+\tan^2\left(4x\right)}\,\mathrm{d}x
		&=\frac19\int \frac{\sec^2\left(4x\right)}{1+\left(\frac{\tan\left(4x\right)}{3}\right)^2}\,\mathrm{d}x
		=\left[\begin{aligned}
			u&=\frac{\tan\left(4x\right)}{3} \\
			\mathrm{d}u&=\frac43\sec^2\left(4x\right)\,\mathrm{d}x
		\end{aligned}\right]
		\frac{1}{9}\cdot\frac{3}{4}\int \frac{1}{1+u^2}\,\mathrm{d}u \\
		&=\frac{1}{12}\arctan\left(u\right)
		=\boxed{\frac{1}{12}\arctan\left(\frac{\tan(4x)}{3}\right)+C}
	\end{align*}
\end{itemize}

\problem{3}
\begin{itemize}
	\item[(b)]
	\begin{align*}
	    \int x^3 \sqrt[3]{x+1}\,\mathrm{d}x
		&=\left[ \begin{aligned}
			u&=x+1 \\
			\mathrm{d}u&=\mathrm{d}x
		\end{aligned} \right]
		\int \left(u-1\right)^3 \sqrt[3]{u}\,\mathrm{d}u
		=\int\left(u^{3+1/3}-3u^{2+1/3}+3u^{1+1/3}-1\cdot u^{1/3}\right)\mathrm{d}u \\
		&=\frac{u^{13/3}}{13/3}-\frac{3u^{10/3}}{10/3}+\frac{3u^{7/3}}{7/3}-\frac{u^{4/3}}{4/3}
		=\frac{3u^4 \sqrt[3]{u}}{13}-\frac{9u^3 \sqrt[3]{u}}{10}+\frac{9u^2 \sqrt[3]{u}}{7}-\frac{3u \sqrt[3]{u}}{4} \\
		&=\boxed{\frac{3(x+1)^4 \sqrt[3]{x+1}}{13}-\frac{9(x+1)^3 \sqrt[3]{x+1}}{10}+\frac{9(x+1)^2 \sqrt[3]{x+1}}{7}-\frac{3(x+1) \sqrt[3]{x+1}}{4}+C}
	\end{align*}
\end{itemize}

\problem{4}
\begin{itemize}
	\item[(b)]
	\begin{align*}
	    \int_{0}^{1/4}\sin^2(\pi x)\,\mathrm{d}x
		&=\frac12 \int_{0}^{1/4}\left(1-\cos\left(2\pi x\right)\right)\mathrm{d}x
		=\left[ \begin{aligned}
			u&=2\pi x \\
			\mathrm{d}u&=2\pi \,\mathrm{d}x
		\end{aligned} \right]
		\frac12 \int_{0}^{1/4}1\,\mathrm{d}x-\frac{1}{2\cdot2\pi} \int_{0}^{\pi/2}\cos(u)\,\mathrm{d}u \\
		&=\frac12 \left[x\right]_{0}^{1/4}-\frac{1}{4\pi}\left[\sin u\right]_{0}^{\pi/2}
		=\boxed{\frac{1}{8}-\frac{1}{4\pi}}
	\end{align*}
\end{itemize}

\problem{5}
\begin{itemize}
	\item[(b)]
	\begin{align*}
	    \int \frac{1}{\sqrt{1+\sqrt{1+x}}}\,\mathrm{d}x
		&=\int \frac{\sqrt{1+\sqrt{1+x}}}{1+\sqrt{1+x}}\,\mathrm{d}x
		=\int \frac{\sqrt{1+\sqrt{1+x}}}{2 \sqrt{1+x}\left(\frac{1}{2 \sqrt{1+x}}+\frac12\right)}\,\mathrm{d}x \\
		&=\left[ \begin{aligned}
			u&=\sqrt{1+x} \\
			\mathrm{d}u&=\frac{\mathrm{d}x}{2 \sqrt{1+x}}
		\end{aligned} \right]
		\int \frac{\sqrt{1+u}}{\frac{1}{2u}+\frac{1}{2}}\,\mathrm{d}u
		=\int \frac{\sqrt{1+u}}{\frac{1+u}{2u}}\,\mathrm{d}u
		=\int \frac{2u\sqrt{1+u}}{1+u}\,\mathrm{d}u \\
		&=2\int \frac{u}{\sqrt{1+u}}\,\mathrm{d}u
		=\left[ \begin{aligned}
			v&=1+u \\
			\mathrm{d}v&=\mathrm{d}u
		\end{aligned} \right]
		2 \int \frac{v-1}{\sqrt{v}}\,\mathrm{d}v
		=2\left(\int v^{1/2}\,\mathrm{d}v-\int v^{-1/2}\,\mathrm{d}v\right) \\
		&=\frac{2v^{3/2}}{3/2}-\frac{2v^{1/2}}{1/2}
		=\frac{4 \sqrt{(1+u)^3}}{3}-4 \sqrt{1+u} \\
		&=\boxed{\frac{4\left(1+\sqrt{1+x}\right)\sqrt{1+\sqrt{1+x}}}{3}-4 \sqrt{1+\sqrt{1+x}}+C}
	\end{align*}
\end{itemize}

\problem{6}
\begin{itemize}
	\item[(a)]
	\begin{align*}
	    \int_{0}^{3}\frac{1}{\sqrt{4-x}}\,\mathrm{d}x
		&=\left[ \begin{aligned}
			u&=4-x \\ 
			\mathrm{d}u&=-\mathrm{d}x
		\end{aligned} \right]
		-\int_{4}^{1}u^{-1/2}\,\mathrm{d}u
		=\left[\frac{u^{1/2}}{1/2}\right]_{4}^{1}
		=2 \sqrt{4}-2 \sqrt{1}
		=\boxed{2}
	\end{align*}

	\item[(b)]
	\begin{align*}
	    \int_{0}^{1}\frac{1}{\sqrt{4-x^2}}\,\mathrm{d}x
		&=\frac12 \int_{0}^{1}\frac{1}{\sqrt{1-\frac{x^2}{4}}}\,\mathrm{d}x
		=\left[ \begin{aligned}
			u&=\frac{x}{2} \\
			\mathrm{d}u&=\frac{\mathrm{d}x}{2}
		\end{aligned} \right]
		\frac12\cdot2 \int_{0}^{1/2}\frac{1}{\sqrt{1-u^2}}\,\mathrm{d}u =\left[\arcsin u\right]_{0}^{1/2}
		=\boxed{\frac{\pi}{6}}
	\end{align*}

	\item[(c)]
	\begin{align*}
	    \int_{0}^{\sqrt{3}}\frac{x}{9+x^4}\,\mathrm{d}x
		&=\frac19\int_{0}^{\sqrt{3}}\frac{x}{1+\frac{x^4}{9}}\,\mathrm{d}x
		=\left[ \begin{aligned}
			u&=\frac{x^2}{3} \\
			\mathrm{d}u&=\frac{2x}{3}\,\mathrm{d}x
		\end{aligned} \right]
		\frac19\cdot \frac{3}{2}\int_{0}^{1}\frac{1}{1+u^2}\,\mathrm{d}u \\
		&=\frac16 \left[\arctan u\right]_{0}^{1}
		=\frac16 \left(\frac{\pi}{4}-0\right)
		=\boxed{\frac{\pi}{24}}
	\end{align*}

	\item[(d)]
	\begin{align*}
		\int_{0}^{1}\frac{x^9}{1+x^{20}}\,\mathrm{d}x
		&=\left[\begin{aligned}
			u&=x^{10} \\
			\mathrm{d}u&=10x^9 \,\mathrm{d}x
		\end{aligned}\right]
		\frac{1}{10}\int_{0}^{1}\frac{1}{1+u^2}\,\mathrm{d}u
		=\frac{1}{10}\left[\arctan u\right]_{0}^{1}
		=\frac{1}{10}\left(\frac{\pi}{4}-0\right)
		=\boxed{\frac{\pi}{40}}
	\end{align*}

	\item[(e)]
	\begin{align*}
		\int_{0}^{2}\frac{2x}{9+x^2}\,\mathrm{d}x
		&=\left[ \begin{aligned}
			u&=9+x^2 \\
			\mathrm{d}u&=2x \,\mathrm{d}x
		\end{aligned} \right]
		\int_{9}^{13}\frac{1}{u}\,\mathrm{d}u
		=\left[\ln u\right]_{9}^{13}
		=\boxed{\ln13-\ln9}
	\end{align*}

	\item[(f)]
	\begin{align*}
		\int \frac{\arctan2x}{1+4x^2}\,\mathrm{d}x
		&=\left[ \begin{aligned}
			u&=\arctan 2x \\
			\mathrm{d}u&=\frac{2}{1+4x^2}\,\mathrm{d}x
		\end{aligned} \right]
		\frac12\int u\,\mathrm{d}u
		=\frac{u^2}{4}
		=\boxed{\frac{\left(\arctan2x\right)^2}{4}+C}
	\end{align*}

	\item[(g)] The function is not continuous because of a discontinuity at $x=\pi^2/4$, so the integral is \boxed{\text{undefined}}.
\end{itemize}

\end{document}
