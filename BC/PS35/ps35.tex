% JUMP TO LINE 60, 73
\documentclass[preview, margin=0.6in]{standalone}
\usepackage[letterpaper,portrait,top=0.4in, left=0.6in, right=0.6in, bottom=1in]{geometry}

\usepackage{amsmath, amsfonts, amsthm, amssymb}
\usepackage{graphicx, float}
\usepackage{mathtools}
\usepackage{titlesec}
\usepackage{interval}
\usepackage{hyperref}
\usepackage{siunitx}
\usepackage{titling}
\usepackage{vwcol}
\usepackage{setspace}
\usepackage{empheq}
\usepackage{cancel}
\usepackage{esdiff}
\usepackage{multicol}
\usepackage{mdframed}
\usepackage{esdiff}
\usepackage{tikzsymbols}
\usepackage{multicol}
\usepackage{tikz}
\usepackage{varwidth}
\usepackage{parskip}
\usepackage{pgfplots}
\pgfplotsset{compat=1.18}
\intervalconfig {
	soft open fences
}

\newcommand{\alignedintertext}[1]{%
  \noalign{%
    \vskip\belowdisplayshortskip
    \vtop{\hsize=\linewidth#1\par
    \expandafter}%
    \expandafter\prevdepth\the\prevdepth
  }%
}

\newtheorem{lemma}{Lemma}

\renewcommand{\qedsymbol}{\Smiley[1.3]}
\newcommand*{\problem}[1]{\section*{Problem #1}}
\newcommand*{\aps}{\section*{AP Corner}}
\newcommand*{\deriv}[1][x]{\ensuremath{\dfrac{\mathrm{d}}{\mathrm{d}#1}}}
\newcommand*{\floor}[1]{\ensuremath{\lfloor #1\rfloor}}
\newcommand*{\lheqzero}{\ensuremath{\underset{\text{L'H}}{\overset{\left[\frac00\right]}{=}}}}
\newcommand*{\lheqinfty}{\ensuremath{\underset{\text{L'H}}{\overset{\left[\frac{\infty}{\infty}\right]}{=}}}}

\DeclareMathOperator{\DNE}{DNE}
\DeclareMathOperator{\sgn}{sgn}

\DeclareMathOperator{\arccsc}{arccsc}
\DeclareMathOperator{\arcsec}{arcsec}
\DeclareMathOperator{\arccot}{arccot}

%opening
\title{\vspace*{-30pt}Problem Set \#35}
\author{Jayden Li}
\date{\today}

% \allowdisplaybreaks
\postdisplaypenalty=100000

\begin{document}
\setstretch{1.25}
\fontsize{12pt}{12pt}\selectfont
\setlength{\abovedisplayskip}{0pt}
\setlength{\parindent}{0pt}
\setlength{\parskip}{2ex plus 0.5ex minus 0.2ex}
\maketitle
\problem{3}
\begin{itemize}
	\item[(a)]
		$\begin{aligned}[t]
			\diff yt=ky \left(1-\frac yK\right)
			&\implies y(0)=\frac{K}{1+Ae^{-kt}}
			=\frac{8\times10^7}{1+\frac{8\times 10^7-2\times10^7}{2\times10^7}e^{-0.71t}}
			=\frac{8\times10^7}{1+3e^{-0.71t}} \\
			&\implies y(1)=\frac{8\times10^7}{1+3e^{-0.71}}
		\end{aligned}$

		Biomass the following year is approximately $3.2\times10^7$ kg.

	\item[(b)]
		$\begin{aligned}[t]
		    \frac{8\times10^7}{1+3e^{-0.71t}}=4\times 10^7
			\implies 1+3e^{-0.71t}=2
			\implies -0.71t=\ln\frac13
			\implies t=1.547
		\end{aligned}$

		It will take approximately $1.547$ years for the biomass to reach $4\times 10^7$ kg.
\end{itemize}

\problem{4}
\begin{itemize}
	\item[(a,b)]
	\textit{I'll derive everything again, because why not}

	$\begin{aligned}[t]
		\diff yt=ky(1-y)
		&\implies \frac{1}{ky(1-y)}\diff yt=1
		\implies \int \frac{1}{ky(1-y)}\,\mathrm{d}y
		\implies \int\left(\frac{A}{ky}+\frac{B}{1-y}\right)\mathrm{d}y=t+C \\
		\alignedintertext{
		\begin{mdframed}
			$\begin{aligned}[t]
			    \frac{A}{ky}+\frac{B}{1-y}=\frac{1}{ky(1-y)}
				&\implies A-Ay+Bky=1
				\implies \left\{\begin{aligned}
						A&=1 \\
						-A+Bk&=0
				\end{aligned}\right.
				\implies A=1 \\
				&\implies -1+Bk=0
				\implies B=1/k
			\end{aligned}$
		\end{mdframed}
		}
		&\implies \int\left(\frac{1}{ky}+\frac{\frac{1}{k}}{1-y}\right)\mathrm{d}y
		=\frac1k \int\left(\frac{1}{y}+\frac{1}{1-y}\right)\mathrm{d}y
		=\frac1k \left(\ln|y|-\ln|1-y|\right) \\
		&\implies \exp(\ln|y|-\ln|1-y|)=\exp(kt+C)
		\implies \frac{|y|}{|1-y|}=Ce^{kt} \\
		&\implies y=Ce^{kt}-Cye^{kt}
		\implies y \left(1+Ce^{kt}\right)=Ce^{kt}
		\implies y=\frac{Ce^{kt}}{1+Ce^{kt}} \\
		&\implies y=\frac{\cancel{Ce^{kt}}}{\cancel{Ce^{kt}}\left(Ce^{-kt}+1\right)}
		\implies \boxed{y=\frac{1}{1+Ce^{-kt}}}
	\end{aligned}$

	where $C$ is a constant determined by the size of the population and the size of the original rumor.

	\item[(c)]
		Let $t$ denote the number of hours since 8am. Then the time at noon is $t=4$.

		$\begin{aligned}[t]
			\frac{1}{1+Ce^{-k\cdot0}}=\frac{80}{1000}
			&\implies \frac{1}{1+C}=\frac{2}{25}
			\implies 2+2C=25
			\implies 2C=23
			\implies C=\frac{23}{2} \\
			\frac{1}{1+Ce^{-k\cdot4}}=\frac12
			&\implies \frac{1}{1+\frac{23}{2}e^{-4k}}=\frac12
			\implies 2+23e^{-4k}=4
			\implies e^{-4k}=\frac{2}{23} \\
			&\implies -4k=\ln \frac{2}{23}
			\implies k=-\frac14\ln \frac{2}{23}
		\end{aligned}$

		$\begin{aligned}[t]
			\frac{1}{1+\frac{23}{2}\exp \left(\frac14t \ln \frac{2}{23}\right)}=\frac{9}{10}
			&\implies 9+\frac{207}{2}\exp \left(\frac t4 \ln \frac{2}{23}\right)=10
			\implies \exp \left(\ln \frac{2}{23}\right)^{t/4}=\frac{2}{207} \\
			&\implies \left(\frac{2}{23}\right)^{t/4}=\frac{2}{207}
			\implies \frac{t}{4}=\log_{2/23}\left(\frac{2}{207}\right)
			\implies t\approx 7.599
		\end{aligned}$

		After $7.599$ hours, or 7 hours and 36 minutes, 90\% of the town would have heard the rumor.
\end{itemize}

\problem{5}
\begin{itemize}
	\item[(a)]
		$\begin{aligned}[t]
		    \diff Pt=kP \left(1-\frac PK\right)
			&\implies \begin{aligned}[t]
				\diff[2] Pt&=k \left(1-\frac PK\right) \diff Pt+kP \left(-\frac1K\diff Pt\right) \\
				&=kP \left(1-\frac PK\right)\left(k \left(1-\frac PK\right)-\frac{kP}{K}\right) \\
				&=k^2P \left(1-\frac PK\right) \left(1-\frac PK-\frac PK\right)
				=k^2P \left(1-\frac PK\right)\left(1-\frac{2P}{k}\right)
			\end{aligned}
		\end{aligned}$

	\item[(b)]
		At the time $t$ when population grows the fastest, $P''(t)=0$.

		$\begin{aligned}[t]
			\left.\diff[2] Pt\right|_{P=k/2}^{}
			&=k^2\cdot\frac k2 \left(1-\frac {\frac k2}K\right)\left(1-\frac{2\cdot\frac k2}{k}\right)
			=\frac{k^3}{2}\left(1-\frac{k}{2K}\right)\left(1-\frac{k}{k}\right)
			=\frac{k^3}{2}\left(1-\frac{k}{2K}\right)\cdot0
			=0
		\end{aligned}$
\end{itemize}

\problem{6}
\begin{itemize}
	\item[(a)]
		$\begin{aligned}[t]
		    \diff Pt=kP-m
			&\implies \int \frac{1}{kP-m}\,\mathrm{d}P=\int \mathrm{d}t
			\implies \frac1k \ln|kP-m|=t+C \\
			&\implies \exp(\ln|kP-m|)=\exp(kt+C)
			\implies kP-m=Ce^{kt}
			\implies P=\frac{Ce^{kt}+m}{k} \\
			P(0)=P_0
			&\implies Ce^{k\cdot0}+m=kP_0
			\implies C=kP_0-m
			\implies\boxed{P(t)=\frac{\left(kP_0-m\right)e^{kt}+m}{k}}
		\end{aligned}$

	\item[(b)] When $kP_0-m>0$ or \boxed{m<kP_0}.
	\item[(c)] Constant population when \boxed{m=kP_0} and declining if \boxed{m>kP_0}.
	\item[(d)] $kP_0=0.016\times8000000=128000<210000=m$ so $m>kP_0$ and the population is declining.
\end{itemize}

\problem{7}
\begin{itemize}
	\item[(a)] 15 fish leave the population ``unnaturally'' (possibly by being caught) every week.
	\item[(b)]
	\item[(c)]
	\item[(d)]
	\item[(e)]
		$\begin{aligned}[t]
		    \diff Pt=0.08P \left(1-\frac{P}{1000}\right)-15
			&\implies \int \frac{1}{0.08P \left(1-\frac{P}{1000}\right)-15}\,\mathrm{d}P=\int \mathrm{d}t=t+C \\
			\text{(by Wolfram Alpha)}&\implies t+C=25\ln(250-P)-25\ln(750-P)
			=25\ln \left(\frac{250-P}{750-P}\right) \\
			&\implies \exp \left(\ln \frac{250-P}{750-P}\right)=\exp \left(\frac{t}{25}+C\right)
			\implies \frac{250-P}{750-P}=Ce^{t/25} \\
			&\implies 250-P=750Ce^{t/25}-CPe^{t/25} \\
			&\implies P\left(Ce^{t/25}-1\right)=750Ce^{t/25}-250
			\implies P=\frac{750Ce^{t/25}-250}{Ce^{t/25}-1} \\
			P(0)=200
			&\implies \frac{750Ce^{0/25}-250}{Ce^{0/25}-1}=200
			\implies \frac{750C-250}{C-1}=200 \\
			&\implies 750C-250=200C-200
			\implies 550C=50
			\implies C=\frac{1}{11} \\
			&\implies P=\frac{\frac{750}{11}e^{t/25}-250}{\frac{1}{11}e^{t/25}-1}
			=\boxed{P=\frac{750e^{t/25}-2750}{e^{t/25}-11}} \\
			P(0)=300
			&\implies \frac{750Ce^{0/25}-250}{Ce^{0/25}-1}=300
			\implies \frac{750C-250}{C-1}=300 \\
			&\implies 750C-250=300C-300
			\implies 450C=-50
			\implies C=-\frac19 \\
			&\implies P=\frac{-\frac{750}{9}e^{t/25}-250}{-\frac{1}{9}e^{t/25}-1}
			=\boxed{P=\frac{750e^{t/25}+2250}{e^{t/25}+9}}
		\end{aligned}$
\end{itemize} 

\end{document}
