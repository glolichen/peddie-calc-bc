% JUMP TO LINE 60, 73
\documentclass{article}
\usepackage[letterpaper,portrait,top=0.4in, left=0.6in, right=0.6in, bottom=1in]{geometry}

\usepackage{amsmath, amsfonts, amsthm, amssymb}
\usepackage{graphicx, float}
\usepackage{mathtools}
\usepackage{titlesec}
\usepackage{interval}
\usepackage{hyperref}
\usepackage{siunitx}
\usepackage{titling}
\usepackage{vwcol}
\usepackage{setspace}
\usepackage{empheq}
\usepackage{cancel}
\usepackage{esdiff}
\usepackage{multicol}
\usepackage{mdframed}
\usepackage{esdiff}
\usepackage{tikzsymbols}
\usepackage{multicol}
\usepackage{tikz}
\usepackage{varwidth}
\usepackage{pgfplots}
\pgfplotsset{compat=1.18}
\intervalconfig {
	soft open fences
}

\newcommand{\alignedintertext}[1]{%
  \noalign{%
    \vskip\belowdisplayshortskip
    \vtop{\hsize=\linewidth#1\par
    \expandafter}%
    \expandafter\prevdepth\the\prevdepth
  }%
}

\newtheorem{lemma}{Lemma}

\renewcommand{\qedsymbol}{\Smiley[1.3]}
\newcommand*{\problem}[1]{\section*{Problem #1}}
\newcommand*{\aps}{\section*{AP Corner}}
\newcommand*{\deriv}[1][x]{\ensuremath{\dfrac{\mathrm{d}}{\mathrm{d}#1}}}
\newcommand*{\floor}[1]{\ensuremath{\lfloor #1\rfloor}}
\newcommand*{\lheqzero}{\ensuremath{\underset{\text{L'H}}{\overset{\left[\frac00\right]}{=}}}}
\newcommand*{\lheqinfty}{\ensuremath{\underset{\text{L'H}}{\overset{\left[\frac{\infty}{\infty}\right]}{=}}}}

\DeclareMathOperator{\DNE}{DNE}
\DeclareMathOperator{\sgn}{sgn}

\DeclareMathOperator{\arccsc}{arccsc}
\DeclareMathOperator{\arcsec}{arcsec}
\DeclareMathOperator{\arccot}{arccot}

\setlength{\parindent}{0pt}

%opening
\title{\vspace*{-30pt}Problem Set \#19}
\author{Jayden Li}
\date{\today}

% \allowdisplaybreaks
\postdisplaypenalty=100000

\begin{document}
\setstretch{1.25}
\fontsize{12pt}{12pt}\selectfont
\setlength{\abovedisplayskip}{0pt}
\maketitle

\problem{2}
\begin{mdframed}
	$\begin{aligned}[t]
		\int \csc x\,\mathrm{d}x
		&=\int \csc(x)\cdot \frac{\csc x+\cot x}{\csc x+\cot x}\,\mathrm{d}x
		=\int \frac{\csc^2x+\csc x\cot x}{\csc x+\cot x}\,\mathrm{d}x \\
		&=\left[\begin{aligned}
				u&=\csc x+\cot x \\
				\mathrm{d}u&=\left(-\csc x\cot x-\csc^2x\right)\mathrm{d}x \\
				&=-\left(\csc^2x+\csc x\cot x\right)\mathrm{d}x \\
		\end{aligned}\right]
		\int \frac{-1}{u}\,\mathrm{d}u
		=\boxed{-\ln|\csc x+\cot x|+C}
	\end{aligned}$
\end{mdframed}
\begin{itemize}
	\item[(b)]
		$\begin{aligned}[t]
		    \int \frac{1}{\sqrt{3}\sin x+\cos x}\,\mathrm{d}x
			&=\int\left(2 \left(\frac{\sqrt{3}}{2}\sin x+\frac12 \cos x\right)\right)^{-1}\mathrm{d}x
			=\frac12\int\left(\cos \frac{\pi}{6}\sin x+\sin \frac{\pi}{6} \cos x\right)^{-1}\mathrm{d}x \\
			&=\frac12 \int\left(\sin \left(x+\frac{\pi}{6}\right)\right)^{-1}\mathrm{d}x
			=\frac12 \int\csc \left(x+\frac{\pi}{6}\right)\,\mathrm{d}x \\
			&=\boxed{-\frac12 \ln \left|\csc \left(x+\frac{\pi}{6}\right)+\cot \left(x+\frac{\pi}{6}\right)\right|+C}
		\end{aligned}$

	\item[(e)]
		$\begin{aligned}[t]
		    \int \frac1x \sqrt{\frac{1-\sqrt{x}}{1+\sqrt{x}}}\,\mathrm{d}x
			&=\left[\begin{aligned}
			    u&=\sqrt{x} \\
				\mathrm{d}u&=\frac{\mathrm{d}x}{2 \sqrt{x}}
				\implies \mathrm{d}x=2u \,\mathrm{d}u
			\end{aligned}\right]
			\int \frac{1}{u^2}\sqrt{\frac{1-u}{1+u}}\cdot2u\,\mathrm{d}u
			=2 \int \frac{\sqrt{1-u}}{u\sqrt{1+u}}\,\mathrm{d}u \\
			&=2 \int \frac{\sqrt{(1-u)(1+u)}}{u(1+u)}\,\mathrm{d}u
			=2 \int \frac{\sqrt{1-u^2}}{u(1+u)}\,\mathrm{d}u \\
			&=\left[\begin{aligned}
					u&=\sin\theta \\
					\mathrm{d}u&=\cos\theta \,\mathrm{d}\theta
			\end{aligned}\right]
			2\int \frac{\sqrt{1-\sin^2\theta}}{\sin(\theta)(1+\sin\theta)}\cdot\cos\theta\,\mathrm{d}\theta
			=2\int \frac{\cos^2\theta}{\sin(\theta)(1+\sin\theta)}\,\mathrm{d}\theta \\
			&=2\int \frac{1-\sin^2\theta}{\sin(\theta)(1+\sin\theta)}\,\mathrm{d}\theta
			=2 \int \frac{1}{\sin(\theta)(1+\sin\theta)}\,\mathrm{d}\theta-2 \int \frac{\sin^2\theta}{\sin(\theta)(1+\sin\theta)}\,\mathrm{d}\theta \\
			&=2 \int \left(\frac{1}{\sin\theta}-\frac{1}{1+\sin\theta}\right)\,\mathrm{d}\theta-2 \int \frac{\sin\theta}{1+\sin\theta}\,\mathrm{d}\theta \\
			&=2 \int\csc\theta\,\mathrm{d}\theta-2\int\frac{1-\sin\theta}{\cos^2\theta}\,\mathrm{d}\theta-2 \int \frac{\sin\theta-\sin^2\theta}{\cos^2\theta}\,\mathrm{d}\theta \\
			&=-2\ln\left|\csc\theta+\cot\theta\right|-2\int \left(\sec^2\theta-\tan\theta\sec\theta\right)\mathrm{d}\theta-2 \int \left(\tan\theta\sec\theta-\tan^2\theta\right)\mathrm{d}\theta \\
			&=-2\ln\left|\csc\theta+\cot\theta\right|-2\tan\theta+2\sec\theta-2\sec\theta+2\tan\theta-2\theta \\
			&=-2\ln\left|\csc\theta+\cot\theta\right|-2\theta
			=-2\ln\left|\frac{1}{u}+\frac{\sqrt{1-u^2}}{u}\right|-2\arcsin u \\
			&=\boxed{-2\ln\left|\frac{1+\sqrt{1-x}}{\sqrt x}\right|-2\arcsin \sqrt x+C}
		\end{aligned}$
\end{itemize}

\problem{3}
$\begin{aligned}
	\int \frac{1}{1+\sin x}\,\mathrm{d}x&=\tan\left(\frac x2+a\right)+b \\
	\deriv[x]\int \frac{1}{1+\sin x}\,\mathrm{d}x&=\deriv[x]\left[\tan\left(\frac x2+a\right)+b\right] \\
	\frac{1}{1+\sin x}&=\sec^2 \left(\frac x2+a\right)\cdot\frac12
	=\frac{1}{\left(\cos\frac x2\cos a-\sin\frac x2\sin a\right)^2}\cdot\frac12 \\
	&=\left(\cos^2\left(\frac{x}{2}\right)\cos^2(a)-2\cos\left(\frac{x}{2}\right)\cos(a)\sin\left(\frac{x}{2}\right)\sin(a)+\sin^2\left(\frac{x}{2}\right)\sin^2(a)\right)^{-1}\cdot\frac12 \\
	1+\sin x&=2\cos^2\left(\frac{x}{2}\right)\cos^2(a)+2\sin^2\left(\frac{x}{2}\right)\sin^2(a)-2\sin(x)\cos(a)\sin(a) \\
\end{aligned}$
\newline

\begin{minipage}[t]{0.59\linewidth}
	\begin{align*}
		2\cos^2\left(\frac{x}{2}\right)\cos^2(a)+2\sin^2\left(\frac{x}{2}\right)\sin^2(a)&=1 \\
		\cos^2 \left(\frac x2\right)\left(1-\sin^2a\right)+\left(1-\cos^2 \left(\frac x2\right)\right)\sin^2a&=\frac12 \\
		\cos^2 \left(\frac x2\right)-2\cos^2 \left(\frac x2\right)\sin^2a+\sin^2a&=\frac12 \\
		\cos^2 \left(\frac x2\right)\left(1-2\sin^2a\right)+\sin^2a&=\frac12
	\end{align*}

	LHS is constant and always equals $1/2$.

	\begin{align*}
		\left\{
			\begin{aligned}
				1-2\sin^2a&=0 \\
				\sin^2a&=\frac12
			\end{aligned}
		\right.
		&\implies
		\left\{
			\begin{aligned}
				\sin^2a&=\frac12 \\
				\sin^2a&=\frac12
			\end{aligned}
		\right. \\
	    \sin^2a=\frac12
		&\implies \sin a=\pm\frac{\sqrt{2}}{2}
	\end{align*}
\end{minipage}
\begin{minipage}[t]{0.4\linewidth}
	\begin{align*}
		-2\sin(x)\cos(a)\sin(a)&=\sin x \\
		\cos(a)\sin(a)&=-\frac12 \\
		\pm\frac{\sqrt{2}}{2}\cos a&=-\frac12 \\
		\cos a&=\mp\frac{\sqrt{2}}{2}
	\end{align*}
	So $a$ is in Quadrant 2 or 4, and $b$ is the constant of integration.

	\begin{equation*}
		\boxed{a=\frac{3\pi}{4}+\pi n,b\in\mathbb R}
	\end{equation*}
\end{minipage}

\problem{4}
\begin{itemize}
	\item[(a)]
		$\begin{aligned}[t]
			\int_{0}^{a}f(a-x)\,\mathrm{d}x
			&=\left[\begin{aligned}
					u&=a-x \\
					\mathrm{d}u&=-\mathrm{d}x
			\end{aligned}\right]
			\int_{a}^{0}(-f(u))\,\mathrm{d}u
			=\int_{0}^{a}f(u)\,\mathrm{d}u
			=\int_{0}^{x}f(x)\,\mathrm{d}x
		\end{aligned}$
\end{itemize}

\end{document}
