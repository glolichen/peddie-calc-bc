\documentclass{article}

\usepackage[letterpaper,portrait,top=0.4in, left=0.6in, right=0.6in, bottom=1in]{geometry}

\usepackage{amsmath, amsfonts, amsthm, amssymb}
\usepackage{graphicx, float}
\usepackage{mathtools}
\usepackage{titlesec}
\usepackage{interval}
\usepackage{hyperref}
\usepackage{siunitx}
\usepackage{titling}
\usepackage{vwcol}
\usepackage{setspace}
\usepackage{empheq}
\usepackage{cancel}
\usepackage{esdiff}
\usepackage{multicol}
\usepackage{mdframed}
\usepackage{esdiff}
\usepackage{tikzsymbols}
\usepackage{multicol}
\usepackage{tikz}
\usepackage{varwidth}

\intervalconfig {
	soft open fences
}

\newcommand{\alignedintertext}[1]{%
  \noalign{%
    \vskip\belowdisplayshortskip
    \vtop{\hsize=\linewidth#1\par
    \expandafter}%
    \expandafter\prevdepth\the\prevdepth
  }%
}

\newtheorem{lemma}{Lemma}

\renewcommand{\qedsymbol}{\Smiley[1.3]}
\newcommand*{\paren}[1]{\ensuremath\left(#1\right)}
\newcommand*{\problem}[1]{\section*{Problem #1}}
\newcommand*{\aps}{\section*{AP Corner}}
\newcommand*{\limit}[2][x]{\ensuremath{\displaystyle\lim_{#1\to#2}}}
\newcommand*{\Limit}[3][x]{\ensuremath{\displaystyle\lim_{#1\to#2}\left[#3\right]}}
\newcommand*{\deriv}[1][x]{\ensuremath{\dfrac{\mathrm{d}}{\mathrm{d}#1}}}
\newcommand*{\Deriv}[2][x]{\ensuremath{\dfrac{\mathrm{d}}{\mathrm{d}#1}\left[#2\right]}}
\newcommand*{\abs}[1]{\ensuremath{\left|#1\right|}}

\newcommand*{\eps}{\varepsilon}

\newcommand*{\floor}[1]{\ensuremath{\lfloor #1\rfloor}}

\DeclareMathOperator{\DNE}{DNE}

%opening
\title{Problem Set \#58}
\author{Jayden Li}
\date{April 3, 2024}

\allowdisplaybreaks

\begin{document}
\setstretch{1.25}
\fontsize{12pt}{12pt}\selectfont
\setlength{\abovedisplayskip}{0pt}
\maketitle

\problem{4}
It is known from Problem 3 that $f'(x)=2x-8$. The slope at $(3,-6)$ is $f'(3)=6-8=-2$. The tangent line is \boxed{y+6=-2(x-3)}.

\problem{5}
\begin{proof}
	If $f$ is continuous, then $\limit{a}f(x)=f(a)\iff\limit{a}f(x)-f(a)=0\iff\Limit{a}{f(x)-f(a)}=0$.
	\begin{equation*}
		\Limit{a}{f(x)-f(a)}=\Limit{a}{\frac{f(x)-f(a)}{x-a}\cdot(x-a)}=\Limit{a}{\frac{f(x)-f(a)}{x-a}}\cdot\Limit{a}{x-a}=f'(x)\cdot0=0
	\end{equation*}
\end{proof}

\problem{7}
\begin{align*}
	\text{velocity}&=\limit[h]{0}\frac{40(2+h)-16(2+h)^2-(40(2)-16(2)^2)}{h} \\
	&=\limit[h]{0}\frac{80+40h-16\paren{4+4h+h^2}-80+64}{h} \\
	&=\limit[h]{0}\frac{80+40h-64-64h-16h^2-80+64}{h} \\
	&=\limit[h]{0}\frac{-24h-16h^2}{h} \\
	&=\Limit[h]{0}{-24-16h} \\
	&=\boxed{-24} \\
\end{align*}

\problem{8}
\begin{align*}
	\diff{s}{t}&=\limit[h]{0}\frac{\frac{1}{(t+h)^2}-\frac{1}{t^2}}{h} \\
	&=\limit[h]{0}\frac{t^2-(t+h)^2}{ht^2(t+h)^2} \\
	&=\limit[h]{0}\frac{t^2-2th-h^2-t^2}{ht^2(t+h)^2} \\
	&=\limit[h]{0}\frac{-2t-h}{t^2(t+h)^2} \\
	&=\frac{\Limit[h]{0}{-2t-h}}{\Limit[h]{0}{t^2(t+h)^2}} \\
	&=\frac{-2t}{t^4} \\
	&=-\frac{2}{t^3}
\end{align*}

\begin{center}
	evaluated at $t=a$: $-2/t^2$

	evaluated at $t=1$: $-2$

	evaluated at $t=2$: $-1/4$

	evaluated at $t=3$: $-2/27$
\end{center}

\problem{10}
\begin{itemize}
	\item[(b)]
	\begin{align*}
		f'(a)&=\limit[h]{0}\frac{\frac{1}{\sqrt{a+h+2}}-\frac{1}{\sqrt{a+2}}}{h} \\
		&=\limit[h]{0}\frac{\sqrt{a+2}-\sqrt{a+h+2}}{h\sqrt{a+h+2}\sqrt{a+2}} \\
		&=\limit[h]{0}\frac{a+2-a-h-2}{h\sqrt{a+h+2}\sqrt{a+2}\paren{\sqrt{a+2}+\sqrt{a+h+2}}} \\
		&=\limit[h]{0}\frac{-1}{\sqrt{a+h+2}\sqrt{a+2}\paren{\sqrt{a+2}+\sqrt{a+h+2}}} \\
		&=\frac{-1}{(a+2)\paren{\sqrt{a+2}+\sqrt{a+2}}} \\
		&=\frac{-1}{2(a+2)\sqrt{a+2}}\cdot\frac{\sqrt{a+2}}{\sqrt{a+2}} \\
		&=\boxed{\frac{-\sqrt{a+2}}{2(a+2)^2}} \\
	\end{align*}
\end{itemize}

\problem{12}
\begin{itemize}
	\item[(a)]
	The rate of change of the price of producing $x$ ounces of gold. Units are dollars per ounce per ounce or dollars per square ounce ($\$/\mathrm{oz}^2$)

	\item[(b)]
	The rate of change at 800 ounces is 17 dollars per square ounces.

	\item[(c)]
	It depends.

	If there is a large amount of gold, then $f'(x)$ will decrease over the long term as the mine can utilize economies of scale.

	If there is not a large amount of gold, then $f'(x)$ will increase over the long term since the cost of producing more gold as the gold runs out is higher.
\end{itemize}

\problem{13}
$T'(10)$ is the rate at which the temperature changes in the neighborhood of $100$ Fahrenheit. I estimate $T'(10)=9$ because the average of $T(10)-T(9)$ and $T(11)-T(10)$ is $9$.

\problem{14}
\begin{itemize}
	\item[(a)]
	The rate of change of solubility in the neighborhood around a certain temperature. Units are milligrams per liter per degree Celsius ($\mathrm{mg/L/\si{\celsius}}$)	

	\item[(b)]
	Approximately $-1/3$? In the neighborhood around $16\si{\celsius}$, the rate of change of solubility is $-1/3\,\mathrm{mg/L/\si{\celsius}}$.
\end{itemize}

\problem{16}
\begin{proof}
	\begin{align*}
		f'(-x)&=\limit[h]{0}\frac{f(-x+h)-f(-x)}{h} \\
		&=\limit[h]{0}\frac{f(-(x-h))-f(-x)}{h} \\
		&=\limit[h]{0}\frac{f(x-h)-f(x)}{h} \\
		&=-\limit[h]{0}\frac{f(x)-f(x-h)}{h} \\
		&=-f'(x)
	\end{align*}
\end{proof}

\end{document}
