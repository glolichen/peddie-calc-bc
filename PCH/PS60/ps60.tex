\documentclass{article}

\usepackage[letterpaper,portrait,top=0.4in, left=0.6in, right=0.6in, bottom=1in]{geometry}

\usepackage{amsmath, amsfonts, amsthm, amssymb}
\usepackage{graphicx, float}
\usepackage{mathtools}
\usepackage{titlesec}
\usepackage{interval}
\usepackage{hyperref}
\usepackage{siunitx}
\usepackage{titling}
\usepackage{vwcol}
\usepackage{setspace}
\usepackage{empheq}
\usepackage{cancel}
\usepackage{esdiff}
\usepackage{multicol}
\usepackage{mdframed}
\usepackage{esdiff}
\usepackage{tikzsymbols}
\usepackage{multicol}
\usepackage{tikz}
\usepackage{varwidth}

\intervalconfig {
	soft open fences
}

\newcommand{\alignedintertext}[1]{%
  \noalign{%
    \vskip\belowdisplayshortskip
    \vtop{\hsize=\linewidth#1\par
    \expandafter}%
    \expandafter\prevdepth\the\prevdepth
  }%
}

\newtheorem{lemma}{Lemma}

\renewcommand{\qedsymbol}{\Smiley[1.3]}
\newcommand*{\paren}[1]{\ensuremath\left(#1\right)}
\newcommand*{\problem}[1]{\section*{Problem #1}}
\newcommand*{\aps}{\section*{AP Corner}}
\newcommand*{\limit}[2][x]{\ensuremath{\displaystyle\lim_{#1\to#2}}}
\newcommand*{\Limit}[3][x]{\ensuremath{\displaystyle\lim_{#1\to#2}\left[#3\right]}}
\newcommand*{\deriv}[1][x]{\ensuremath{\dfrac{\mathrm{d}}{\mathrm{d}#1}}}
\newcommand*{\Deriv}[2][x]{\ensuremath{\dfrac{\mathrm{d}}{\mathrm{d}#1}\left[#2\right]}}
\newcommand*{\abs}[1]{\ensuremath{\left|#1\right|}}

\newcommand*{\eps}{\varepsilon}

\newcommand*{\floor}[1]{\ensuremath{\lfloor #1\rfloor}}

\DeclareMathOperator{\DNE}{DNE}

%opening
\title{Problem Set \#60}
\author{Jayden Li}
\date{April 6, 2024}

\allowdisplaybreaks

\begin{document}
\setstretch{1.25}
\fontsize{12pt}{12pt}\selectfont
\setlength{\abovedisplayskip}{0pt}
\maketitle

\problem{1}
\begin{proof}
	\begin{align*}
		f'(x)&=\limit[h]{0}\frac{f(x+h)-f(x)}{h} \\
		&=\limit[h]{0}\frac{c-c}{h} \\
		&=\limit[h]{0}\frac{0}{h} \\
		&=0
	\end{align*}
\end{proof}

\problem{2}
\begin{proof}
	Let $f_n(x)=x^n$, where $n\in\mathbb{Z}^+$.

	\textbf{Base case.} $n=1$.
	\begin{equation*}
		\deriv x=1=1\cdot1=1x^{1-1}
	\end{equation*}

	\textbf{Hypothesis.} Suppose that $f_n'(x)=nx^{n-1}$. NTS $f_{n+1}'(x)=(n+1)x^n$.

	\textbf{Inductive step.}
	\begin{align*}
		f_{n+1}'(x)&=\Deriv{x^{n+1}} \\
		&=\Deriv{x\cdot x^n} \\
		&=\Deriv{x}\cdot x^n+x\cdot\Deriv{x^n} \\
		&=x^n+x\cdot\paren{nx^{n-1}} \\
		&=x^n+nx^n \\
		&=(n+1)x^n
	\end{align*}
\end{proof}

\problem{3}
\begin{multicols}{2}
	\allowdisplaybreaks[0]
	\begin{itemize}
		\item[(a)]
		\begin{align*}
			f'(x)&=\Deriv{\sin x+\sin\paren{x^2}} \\
			&=\boxed{\cos x+2x\cos\paren{x^2}}
		\end{align*}

		\item[(b)]
		\begin{align*}
			f'(x)&=\deriv\sin(\sin x) \\
			&=\boxed{\cos(\sin x)\cos x}
		\end{align*}

		\item[(c)]
		\begin{align*}
			f'(x)&=\Deriv{\frac{\sin(\cos x)}{x}} \\
			&=\frac{x\cos(\cos x)(-\sin x)-\sin(\cos x)}{x^2} \\
			&=\boxed{\frac{-x\cos(\cos x)\sin x-\sin(\cos x)}{x^2}}
		\end{align*}

		\item[(d)]
		\begin{align*}
			f'(x)&=\Deriv{\sin(\cos(\sin x))} \\
			&=\cos(\cos(\sin x))\cdot\Deriv{\cos(\sin x)} \\
			&=\cos(\cos(\sin x))\cdot(-\sin(\sin x)\cdot\cos x) \\
			&=\boxed{-\cos(\cos(\sin x))\sin(\sin x)\cos x}
		\end{align*}
	\end{itemize}
	\allowdisplaybreaks
\end{multicols}

\problem{4}
\begin{itemize}
	\item[(a)]
	\begin{align*}
		f'(x)&=\Deriv{\sin^3\paren{x^2+\sin x}} \\
		&=\Deriv{\paren{\sin\paren{x^2+\sin x}}^3} \\
		&=3\paren{\sin\paren{x^2+\sin x}}^2\cdot\Deriv{\sin\paren{x^2+\sin x}} \\
		&=\boxed{3\paren{\sin\paren{x^2+\sin x}}^2\cdot\cos\paren{x^2+\sin x}\cdot\paren{2x+\cos x}} \\
	\end{align*}

	\item[(b)]
	\begin{align*}
		f'(x)&=\Deriv{\sin\paren{\frac{x^3}{\cos\paren{x^3}}}} \\
		&=\cos\paren{\frac{x^3}{\cos\paren{x^3}}}\cdot\Deriv{\frac{x^3}{\cos\paren{x^3}}} \\
		&=\cos\paren{\frac{x^3}{\cos\paren{x^3}}}\cdot\frac{3x^2\cos\paren{x^3}-x^3\paren{-\sin\paren{x^3}\cdot3x^2}}{\cos^2\paren{x^3}} \\
		&=\boxed{\cos\paren{\frac{x^3}{\cos\paren{x^3}}}\cdot\frac{3x^2\cos\paren{x^3}+3x^5\sin\paren{x^3}}{\cos^2\paren{x^3}}} \\
	\end{align*}

	\item[(c)]
	\begin{align*}
		f'(x)&=\Deriv{(\cos x)^{31^2}} \\
		&=\Deriv{(\cos x)^{961}} \\
		&=\boxed{-961\cos^{960}(x)\sin(x)}
	\end{align*}
\end{itemize}

\problem{5}
\begin{multicols}{2}
	\allowdisplaybreaks[0]
	\begin{itemize}
		\item[(a)]
		\begin{align*}
			f'(x)&=\cos x \\
			f'(f(x))&=\boxed{\cos(\sin x)}
		\end{align*}
		
		\item[(b)]
		\begin{align*}
			f'(x)&=0 \\
			f'(f(x))&=\boxed{0}
		\end{align*}
	\end{itemize}
	\allowdisplaybreaks
\end{multicols}

\problem{6}
\begin{multicols}{2}
	\allowdisplaybreaks[0]
	\begin{itemize}
		\item[(a)]
		\begin{align*}
			f'(x)&=\Deriv{g(xg(a))} \\
			&=g'(xg(a))\cdot\Deriv{xg(a)} \\
			&=\boxed{g'(xg(a))\paren{g(a)+xg'(a)}} \\
		\end{align*}

		\item[(b)]
		\begin{align*}
			f'(x)&=\Deriv{g(x)(x-a)}\\
			&=\Deriv{xg(x)}-\Deriv{ag(x)} \\
			&=\boxed{g(x)+xg'(x)-ag'(x)}
		\end{align*}

		\item[(c)]
		Let $y=x+3$.
		\begin{align*}
			f(y)&=g\paren{(y-3)^2} \\
			f'(y)&=g'\paren{(y-3)^2}\cdot2(y-3)\cdot1 \\
			\Aboxed{f'(y)&=2g'\paren{(y-3)^2}(y-3)}
		\end{align*}
	\end{itemize}
	\allowdisplaybreaks
\end{multicols}

\problem{7}
Let $f(x)$ and $g(x)$ be the radii of the larger and smaller circle, respectively, then:
\begin{align*}
	\pi(f(x))^2-\pi(g(x))^2&=9\pi \\
	\pi(g(x))^2&=\pi(f(x))^2-9\pi \\
	(g(x))^2&=(f(x))^2-9 \\
	\Deriv{(g(x))^2}&=\Deriv{(f(x))^2-9} \\
	2g(x)g'(x)&=2f(x)f'(x) \\
	g'(x)&=\frac{f(x)f'(x)}{g(x)} \tag{1}
\end{align*}

We need to calculate $g'$ at $x=t$. It is known that the circumference of the smaller circle is $16\pi$, so $2g(t)\pi=16\pi$, thus $g(t)=8$. Also, the rate of change of the area of the larger circle is $10\pi$.
\begin{align*}
	\Deriv{\pi(f(x))^2}&=10\pi \\
	\pi\cdot2f(x)f'(x)&=10\pi \\
	f(x)f'(x)&=5 \\
	f(t)f'(t)&=5
\end{align*}

Substituting into (1), we have:
\begin{align*}
	g'(x)&=\frac{f(x)f'(x)}{g(x)} \tag{1} \\
	g'(t)&=\frac{f(t)f'(t)}{g(t)} \\
	&=\boxed{\frac{5}{8}}
\end{align*}

\problem{8}
\begin{align*}
	f'(x)&=\Deriv{x^2\sin\paren{\frac{1}{x}}} \\
	&=2x\sin\paren{\frac{1}{x}}+x^2\cos\paren{\frac{1}{x}}\cdot\paren{-\cdot\frac{1}{x^2}} \\
	&=2x\sin\paren{\frac{1}{x}}-\cos\paren{\frac{1}{x}}
\end{align*}
$f$ and $f'$ are both undefined at $x=0$.

\begin{itemize}
	\item[(a)]
	\begin{align*}
		(f\circ h)'(0)&=f'(h(0))\cdot h'(0) \\
		&=f'(0)\cdot\sin^2(\sin(1))
	\end{align*}
	\boxed{\text{Undefined.}}

	\item[(b)]
	\begin{align*}
		(k\circ f)'(0)&=k'(f(0))\cdot f'(0)
	\end{align*}
	\boxed{\text{Undefined.}}
\end{itemize}

\problem{9}
\begin{itemize}
	\item[(a)]
	\begin{align*}
		f'(x)&=\deriv\sqrt{1-x^2} \\
		&=\frac{1}{2\sqrt{1-x^2}}\cdot\paren{0-2x} \\
		&=\boxed{-\frac{x}{\sqrt{1-x^2}}}
	\end{align*}

	\item[(b)]
	\begin{proof}
		\begin{align*}
			y-\sqrt{1-a^2}&=-\frac{a}{\sqrt{1-a^2}}(x-a) \\
			y&=-\frac{a}{\sqrt{1-a^2}}(x-a)+\sqrt{1-a^2} \\
			\sqrt{1-x^2}&=\frac{-ax}{\sqrt{1-a^2}}+\frac{a^2}{\sqrt{1-a^2}}+\frac{1-a^2}{\sqrt{1-a^2}} \\
			\sqrt{1-x^2}&=\frac{1-ax}{\sqrt{1-a^2}} \\
			1-x^2&=\frac{1-2ax+a^2x^2}{1-a^2} \\
			1-a^2-x^2+a^2x^2&=1-2ax+a^2x^2 \\
			-a^2-x^2&=-2ax \\
			x^2-2ax+a^2&=0 \\
			x&=\frac{2a\pm\sqrt{4a^2-4a^2}}{2} \\
			x&=\frac{2a}{2} \\
			x&=a
		\end{align*}
		The line intersects the curve at only one point, so it is a tangent line.
	\end{proof}
\end{itemize}

\problem{10}
\begin{itemize}
	\item[(a)]
	\begin{equation*}
		g'=2f\cdot f'
	\end{equation*}

	\item[(b)]
	\begin{equation*}
		g'=2f'\cdot f''
	\end{equation*}

	\item[(c)]
	\begin{align*}
		(f')^2=f+\frac{1}{f^3} \\
		\Deriv{(f')^2}&=\Deriv{f}+\Deriv{\frac{1}{f^3}} \\
		2f'\cdot f''&=f'+\paren{\frac{-3}{f^4}}\cdot f' \\
		\Aboxed{f''&=\frac{f'-\frac{3f'}{f^4}}{2f'}}
	\end{align*}
\end{itemize}

\problem{11}
\begin{proof}
	By induction

	\textbf{Base case.} $k=1$.
	\begin{equation*}
		f'(x)=-nx^{-n-1}=(-1)^1\cdot n\cdot x^{-n-1}=(-1)^1\cdot\frac{n\cdot(n-1)!}{(n-1)!}x^{-n-1}=(-1)^1\frac{(n+1-1)!}{(n-1)!}x^{-n-1}
	\end{equation*}

	\textbf{Hypothesis.} Suppose that
	\begin{equation*}
		f^{(k)}(x)=(-1)^k\frac{(n+k-1)!}{(n-1)!}x^{-n-k}
	\end{equation*}

	NTS:
	\begin{align*}
		f^{(k+1)}(x)&=(-1)^{k+1}\frac{(n+k+1-1)!}{(n-1)!}x^{-n-(k+1)} \\
		&=-(-1)^k\frac{(n+k)!}{(n-1)!}x^{-n-k-1}
	\end{align*}

	\textbf{Inductive step.}
	\begin{align*}
		f^{(k+1)}(x)&=\Deriv{f^{(k)}(x)} \\
		&=\Deriv{(-1)^k\frac{(n+k-1)!}{(n-1)!}x^{-n-k}} \\
		&=(-1)^k\frac{(n+k-1)!}{(n-1)!}\cdot\deriv x^{-n-k} \\
		&=(-1)^k\frac{(n+k-1)!}{(n-1)!}\cdot(-n-k)\cdot x^{-n-k-1} \\
		&=-(-1)^k\frac{(n+k)\cdot(n+k-1)!}{(n-1)!}\cdot x^{-n-k-1} \\
		&=-(-1)^k\frac{(n+k)!}{(n-1)!}x^{-n-k-1}
	\end{align*}
\end{proof}

\end{document}
