\documentclass{article}

\usepackage[letterpaper,portrait,top=0.4in, left=0.6in, right=0.6in, bottom=1in]{geometry}

\usepackage{amsmath, amsfonts, amsthm, amssymb}
\usepackage{graphicx, float}
\usepackage{mathtools}
\usepackage{titlesec}
\usepackage{interval}
\usepackage{hyperref}
\usepackage{siunitx}
\usepackage{titling}
\usepackage{vwcol}
\usepackage{setspace}
\usepackage{empheq}
\usepackage{cancel}
\usepackage{esdiff}
\usepackage{multicol}
\usepackage{mdframed}
\usepackage{esdiff}
\usepackage{tikzsymbols}
\usepackage{multicol}
\usepackage{tikz}
\usepackage{varwidth}
\usepackage{pgfplots}

\intervalconfig {
	soft open fences
}

\newcommand{\alignedintertext}[1]{%
  \noalign{%
    \vskip\belowdisplayshortskip
    \vtop{\hsize=\linewidth#1\par
    \expandafter}%
    \expandafter\prevdepth\the\prevdepth
  }%
}

\newtheorem{lemma}{Lemma}

\renewcommand{\qedsymbol}{\Smiley[1.3]}
\newcommand*{\paren}[1]{\ensuremath\left(#1\right)}
\newcommand*{\problem}[1]{\section*{Problem #1}}
\newcommand*{\aps}{\section*{AP Corner}}
\newcommand*{\limit}[2][x]{\ensuremath{\displaystyle\lim_{#1\to#2}}}
\newcommand*{\Limit}[3][x]{\ensuremath{\displaystyle\lim_{#1\to#2}\left[#3\right]}}
\newcommand*{\deriv}[1][x]{\ensuremath{\dfrac{\mathrm{d}}{\mathrm{d}#1}}}
\newcommand*{\Deriv}[2][x]{\ensuremath{\dfrac{\mathrm{d}}{\mathrm{d}#1}\left[#2\right]}}
\newcommand*{\iinteg}[2][x]{\ensuremath{\displaystyle\int #2\;\mathrm{d}#1}}
\newcommand*{\dinteg}[4][x]{\ensuremath{\displaystyle\int_{#2}^{#3}#4\;\mathrm{d}#1}}
\newcommand*{\abs}[1]{\ensuremath{\left|#1\right|}}
\newcommand*{\eps}{\varepsilon}
\newcommand*{\floor}[1]{\ensuremath{\lfloor #1\rfloor}}
\newcommand*{\cbrt}[1]{\ensuremath{\sqrt[3]{#1}}}

\DeclareMathOperator{\DNE}{DNE}

%opening
\title{Problem Set \#64}
\author{Jayden Li}
\date{April 22, 2024}

\allowdisplaybreaks

\begin{document}
\setstretch{1.25}
\fontsize{12pt}{12pt}\selectfont
\setlength{\abovedisplayskip}{0pt}
\maketitle

\problem{1}
\begin{equation*}
	f'(x)=12x^3-12x^2-24x
\end{equation*}
\begin{multicols}{2}
	Increasing:
	\begin{align*}
		12x^3-12x^2-24x&>0 \\
		12\paren{x^2-x-2}&>0
	\end{align*}

	\textbf{C1:} $x=0$. Then $12x\paren{x^2-x-2}=0\not>0$.

	\textbf{C2:} $x>0$.
	\begin{align*}
		x^2-x-2&>0 \\
		(x-2)(x+1)&>0
	\end{align*}.
	\begin{tikzpicture}	
		\draw
		(0,0) node[circle,draw,inner sep=1pt,label=below:$-\infty$](){}
	 -- (2.5,0) node[circle,draw,inner sep=2pt,label=below:$-1$](){} node[midway,above]{$+$}
	 -- (4,0) node[circle,draw,inner sep=2pt,label=below:$2$](){} node[midway,above]{$-$}
	 -- (6,0) node[circle,draw,inner sep=1pt,label=below:$\infty$](){} node[midway,above]{$+$};
	\end{tikzpicture}
	\begin{align*}
		x&\in((-\infty,-1)\cup(2,\infty))\cap(0,\infty) \\
		x&\in(2,\infty)
	\end{align*}

	\textbf{C3:} $x<0$.
	\begin{align*}
		x^2-x-2&<0 \\
		(x-2)(x+1)&<0
	\end{align*}.
	\begin{tikzpicture}	
		\draw
		(0,0) node[circle,draw,inner sep=1pt,label=below:$-\infty$](){}
	 -- (2.5,0) node[circle,draw,inner sep=2pt,label=below:$-1$](){} node[midway,above]{$+$}
	 -- (4,0) node[circle,draw,inner sep=2pt,label=below:$2$](){} node[midway,above]{$-$}
	 -- (6,0) node[circle,draw,inner sep=1pt,label=below:$\infty$](){} node[midway,above]{$+$};
	\end{tikzpicture}
	\begin{align*}
		x&\in(-1,2)\cap(-\infty,0) \\
		x&\in(-1,0)
	\end{align*}

	\begin{equation*}
		\boxed{x\in(-1,0)\cup(2,\infty)}
	\end{equation*}

	\columnbreak

	Decreasing:
	\begin{align*}
		12x^3-12x^2-24x&<0 \\
		12\paren{x^2-x-2}&<0
	\end{align*}

	\textbf{C1:} $x=0$. Then $12x\paren{x^2-x-2}=0\not<0$.

	\textbf{C2:} $x>0$.
	\begin{align*}
		x^2-x-2&<0 \\
		(x-2)(x+1)&<0
	\end{align*}.
	\begin{tikzpicture}	
		\draw
		(0,0) node[circle,draw,inner sep=1pt,label=below:$-\infty$](){}
	 -- (2.5,0) node[circle,draw,inner sep=2pt,label=below:$-1$](){} node[midway,above]{$+$}
	 -- (4,0) node[circle,draw,inner sep=2pt,label=below:$2$](){} node[midway,above]{$-$}
	 -- (6,0) node[circle,draw,inner sep=1pt,label=below:$\infty$](){} node[midway,above]{$+$};
	\end{tikzpicture}
	\begin{align*}
		x&\in(-1,2)\cap(0,\infty) \\
		x&\in(0,2)
	\end{align*}

	\textbf{C3:} $x<0$.
	\begin{align*}
		x^2-x-2&>0 \\
		(x-2)(x+1)&>0
	\end{align*}.
	\begin{tikzpicture}	
		\draw
		(0,0) node[circle,draw,inner sep=1pt,label=below:$-\infty$](){}
	 -- (2.5,0) node[circle,draw,inner sep=2pt,label=below:$-1$](){} node[midway,above]{$+$}
	 -- (4,0) node[circle,draw,inner sep=2pt,label=below:$2$](){} node[midway,above]{$-$}
	 -- (6,0) node[circle,draw,inner sep=1pt,label=below:$\infty$](){} node[midway,above]{$+$};
	\end{tikzpicture}
	\begin{align*}
		x&\in((-\infty,-1)\cup(2,\infty))\cap(-\infty,0) \\
		x&\in(-\infty,-1)
	\end{align*}

	\begin{equation*}
		\boxed{x\in(\infty,-1)\cup(0,2)}
	\end{equation*}
\end{multicols}

\pagebreak
\problem{2}
\begin{itemize}
	\item[(a)]
	Increasing: $(1,5)$.

	Decreasing: $[0,1)\cup(5,6]$.

	\item[(b)]
	$x\in\{1,5\}$.
\end{itemize}

\problem{3}
Local maximum: $x=1$. Local maximum: $x=5$.

\problem{4}
\begin{align*}
	g(x)&=x+2\sin x \\
	g'(x)&=1+2\cos x \\
	g''(x)&=-2\sin x
\end{align*}
Critical points:
\begin{align*}
	g'(x)&=0 \\
	1+2\cos x&=0 \\
	\cos x&=-\frac{1}{2} \\
	x&=\pm\frac{2\pi}{3}+2\pi n
	\intertext{Use second derivative test:}
	g''\paren{\frac{2\pi}{3}+2\pi n}&=-2\sin\paren{\frac{2\pi}{3}+2\pi n} \\
	&=-2\cdot\paren{\frac{\sqrt{3}}{2}} \\
	&=-\sqrt{3}
	\intertext{Local maxima: \boxed{\dfrac{2\pi}{3}+2\pi n}}
	g''\paren{-\frac{2\pi}{3}+2\pi n}&=-2\sin\paren{-\frac{2\pi}{3}+2\pi n} \\
	&=-2\cdot\paren{-\frac{\sqrt{3}}{2}} \\
	&=\sqrt{3}
	\intertext{Local minima: \boxed{-\dfrac{2\pi}{3}+2\pi n}}
\end{align*}

\problem{5}
\begin{itemize}
	\item[(i)]
	

	\item[(ii)]
	

	\item[(iii)]
	

\end{itemize}

\problem{6}
\begin{itemize}
	\item[(i)]
	\begin{align*}
		f(x)&=\sin x+\cos x \\
		f'(x)&=\cos x-\sin x \\
		f''(x)&=-\sin x-\cos x
	\end{align*}
	\begin{itemize}
		\item[(a)]
		\begin{multicols}{2}
			Increasing:
			\begin{align*}
				f'(x)&>0 \\
				\cos x-\sin x&>0 \\
				\cos x&>\sin x \\
				\Aboxed{x&\in\paren{0,\frac{\pi}{4}}\cup\paren{\frac{5\pi}{4},2\pi}}
			\end{align*}

			\columnbreak

			Decreasing:
			\begin{align*}
				f'(x)&<0 \\
				\cos x-\sin x&<0 \\
				\cos x&<\sin x \\
				\Aboxed{x&\in\paren{\frac{\pi}{4},\frac{5\pi}{4}}}
			\end{align*}
		\end{multicols}
		(I figured out the answers by looking at a unit circle.)

		\item[(b)]
		\begin{align*}
			f'(x)&=0 \\
			\cos x-\sin x&=0 \\
			\cos x&=\sin x \\
			x&\in\left\{\frac{\pi}{4},\frac{5\pi}{4}\right\} \\
			f''\paren{\frac{\pi}{4}}&=-\sin\paren{\frac{\pi}{4}}-\cos\paren{\frac{\pi}{4}} \\
			&=-\frac{\sqrt{2}}{2}-\frac{\sqrt{2}}{2} \\
			&=-\sqrt{2}<0 \\
			f''\paren{\frac{5\pi}{4}}&=-\sin\paren{\frac{5\pi}{4}}-\cos\paren{\frac{5\pi}{4}} \\
			&=\frac{\sqrt{2}}{2}-\paren{-\frac{\sqrt{2}}{2}} \\
			&=\sqrt{2}>0
		\end{align*}
		Local maximum: \boxed{x=\dfrac{5\pi}{4}}. Local minimum: \boxed{x=\dfrac{\pi}{4}}
	
		\item[(c)]
		\begin{multicols}{2}
			Concave up:
			\begin{align*}
				f''(x)&>0 \\
				-\sin x-\cos x&>0 \\
				-\sin x&>\cos x \\
				\Aboxed{x&\in\paren{\frac{3\pi}{4},\frac{7\pi}{4}}}
			\end{align*}

			\columnbreak

			Concave down:
			\begin{align*}
				f''(x)&<0 \\
				-\sin x-\cos x&<0 \\
				-\sin x&<\cos x \\
				\Aboxed{x&\in\paren{0,\frac{3\pi}{4}}\cup\paren{\frac{7\pi}{4},2\pi}}
			\end{align*}
		\end{multicols}
		Inflection points:
		\begin{align*}
			f''(x)&=0 \\
			-\sin x-\cos x&=0 \\
			-\sin x&=\cos x \\
			\Aboxed{x&\in\left\{\frac{3\pi}{4},\frac{7\pi}{4}\right\}}
		\end{align*}
	\end{itemize}

	\item[(ii)]
	\begin{align*}
		f(x)&=e^{2x}+e^{-x} \\
		f'(x)&=2e^{2x}-e^{-x} \\
		f''(x)&=4e^{2x}+e^{-x}
	\end{align*}
	\begin{itemize}
		\item[(a)]
		\begin{multicols}{2}
			Increasing:
			\begin{align*}
				f'(x)&>0 \\
				2e^{2x}-e^{-x}&>0 \\
				e^x\paren{2e^{2x}-\frac{1}{e^x}}&>0
				\intertext{($e^x$ is always positive.)}
				2e^{3x}&>1 \\
				e^{3x}&>\frac{1}{2} \\
				3x&>\ln\paren{\frac{1}{2}} \\
				\intertext{(natural logarithm is an increasing function.)}
				x&>\frac{\ln(0.5)}{3} \\
				\Aboxed{x&\in\paren{\frac{\ln(0.5)}{3},\infty}}
			\end{align*}

			\columnbreak

			Decreasing:
			\begin{align*}
				f'(x)&<0 \\
				2e^{2x}-e^{-x}&<0 \\
				e^x\paren{2e^{2x}-\frac{1}{e^x}}&<0 \\
				2e^{3x}&<1 \\
				e^{3x}&<\frac{1}{2} \\
				3x&<\ln\paren{\frac{1}{2}} \\
				x&<\frac{\ln(0.5)}{3} \\
				\Aboxed{x&\in\paren{-\infty,\frac{\ln(0.5)}{3}}}
			\end{align*}
		\end{multicols}

		\item[(b)]
		\begin{align*}
			f'(x)&=0 \\
			2e^{2x}-e^{-x}&=0 \\
			2e^{3x}&=1 \\
			e^{3x}&=\frac{1}{2} \\
			x&=\frac{\ln(0.5)}{3} \\
			f''\paren{\frac{\ln(0.5)}{3}}&=4\exp\paren{\frac{2\ln(0.5)}{3}}+\exp\paren{-\frac{\ln(0.5)}{3}}>0
		\end{align*}
		(since the exponential function is always positive, the sum of two exp's must be positive.)

		Local minimum: \boxed{x=\dfrac{\ln(0.5)}{3}}
	
		\item[(c)]
		\begin{multicols}{2}
			Concave up:
			\begin{align*}
				f''(x)&>0 \\
				4e^{2x}+e^{-x}&>0 \\
				4e^{3x}&>-1
			\end{align*}
			Which is never true for real values of $x$.

			\columnbreak

			Concave down:
			\begin{align*}
				f''(x)&<0 \\
				4e^{2x}+e^{-x}&<0 \\
				4e^{3x}&<-1 \\
				\Aboxed{x&\in\mathbb{R}}
			\end{align*}
		\end{multicols}
		Inflection points:
		\begin{align*}
			f''(x)&=0 \\
			4e^{2x}+e^{-x}&=0 \\
			4e^{3x}&=-1
		\end{align*}
		Never true $\implies$ no inflection points.
	\end{itemize}

	\item[(iii)]
	\begin{align*}
		f(x)&=\frac{\ln x}{\sqrt{x}} \\
		f'(x)&=\frac{\frac{1}{x}\cdot\sqrt{x}-\ln(x)\cdot\frac{1}{2\sqrt{x}}}{\paren{\sqrt{x}}^2}=\frac{\frac{\sqrt{x}}{x}-\frac{\sqrt{x}\ln x}{2x}}{x}=\frac{2\sqrt{x}-\sqrt{x}\ln x}{2x^2} \\
		f''(x)&=\frac{\paren{\frac{2}{2\sqrt{x}}-\frac{\ln x}{2\sqrt{x}}-\frac{\sqrt{x}}{x}}2x^2-\paren{2\sqrt{x}-\sqrt{x}\ln x}4x}{4x^4}
		=\frac{\frac{2x^2}{\sqrt{x}}-\frac{2x^2\ln x}{2\sqrt{x}}-\frac{2x^2}{\sqrt{x}}-\frac{8x^2}{\sqrt{x}}+\frac{4x^2\ln x}{\sqrt{x}}}{4x^4} \\
		&=\frac{4x^2-2x^2\ln x-4x^2-16x^2+8x^2\ln x}{8x^4\sqrt{x}}
		=\frac{-\ln x-8+4\ln x}{4x^2\sqrt{x}}=\frac{3\ln x-8}{4x^2\sqrt{x}}
	\end{align*}
	\begin{itemize}
		\item[(a)]
		\begin{multicols}{2}
			Increasing:
			\begin{align*}
				f'(x)&>0 \\
				\frac{2\sqrt{x}-\sqrt{x}\ln x}{2x^2}&>0 \\
				2\sqrt{x}&>\sqrt{x}\ln x \\
				2&>\ln x \\
				\Aboxed{x&\in(0,e^2)}
			\end{align*}
			($2x^2$ and $\sqrt{x}$ must be positive and natural logarithm is increasing, and domain is $(0,\infty)$)

			\columnbreak

			Decreasing:
			\begin{align*}
				f'(x)&<0 \\
				\frac{2\sqrt{x}-\sqrt{x}\ln x}{2x^2}&<0 \\
				2\sqrt{x}&<\sqrt{x}\ln x \\
				2&<\ln x \\
				\Aboxed{x&\in(e^2,\infty)}
			\end{align*}
		\end{multicols}
	
		\pagebreak
		\item[(b)]
		\begin{align*}
			f'(x)&=0 \\
			\frac{2\sqrt{x}-\sqrt{x}\ln x}{2x^2}&=0 \\
			2\sqrt{x}&=\sqrt{x}\ln x \\
			2&=\ln x \\
			x&=e^2 \\
			f''(e^2)&=\frac{3\ln e^2-8}{4\paren{e^2}^2\sqrt{e^2}} \\
			&=\frac{3\cdot2-8}{4e^4\cdot e} \\
			&=-\frac{1}{2e^5}<0
		\end{align*}
		Local maximum: \boxed{x=e^2}
	
		\item[(c)]
		\begin{multicols}{2}
			Concave up:
			\begin{align*}
				f''(x)&>0 \\
				\frac{3\ln x-8}{4x^2\sqrt{x}}&>0 \\
				3\ln x&>8
				\intertext{(denominator is always positive)}
				x&>e^{8/3} \\
				\Aboxed{x&\in\paren{e^{8/3},\infty}}
			\end{align*}

			\columnbreak

			Concave down:
			\begin{align*}
				f''(x)&<0 \\
				\frac{3\ln x-8}{4x^2\sqrt{x}}&<0 \\
				3\ln x&<8 \\
				x&<e^{8/3} \\
				\Aboxed{x&\in\paren{0,e^{8/3}}}
			\end{align*}
		\end{multicols}
		Inflection points:
		\begin{align*}
			f''(x)&=0 \\
			\frac{3\ln x-8}{4x^2\sqrt{x}}&=0 \\
			3\ln x&=8 \\
			\Aboxed{x&=e^{8/3}}
		\end{align*}
	\end{itemize}
\end{itemize}

\problem{7}
\begin{itemize}
	\item[(a)]
	

	\item[(b)]
	

\end{itemize}

\problem{8}
\begin{itemize}
	\item[(a)]
	

	\item[(b)]
	
	
	\item[(c)]
	

	\item[(d)]
	

	\item[(e)]
	

\end{itemize}


\problem{9}


\problem{10}
\begin{itemize}
	\item[(a)]
	

	\item[(b)]
	

	\item[(c)]
	
	
\end{itemize}

\problem{11}


\problem{12}


\problem{13}


\problem{14}
\begin{align*}
	y''&=0 \\
	\Deriv{\frac{\paren{1+x^2}-2x(1+x)}{\paren{1+x^2}^2}}&=0 \\
	\Deriv{\frac{1-x^2-2x}{\paren{1+x^2}^2}}&=0 \\
	\frac{\paren{-2x-2}\paren{1+x^2}^2-\paren{1-x^2-2x}\paren{2\paren{1+x^2}}\paren{2x}}{\paren{1+x^2}^4}&=0 \\
	\paren{-2x-2}\paren{1+x^2}-\paren{1-x^2-2x}\paren{2}\paren{2x}&=0 \\
	-2x-2x^3-2-2x^2-4x\paren{1-x^2-2x}&=0 \\
	-2x-2x^3-2-2x^2-4x+4x^3+8x^2&=0 \\
	2x^3+6x^2-6x-2&=0 \\
	x^3+3x^2-3x-1&=0 \\
	(x-1)\paren{x^2+4x+1}&=0 \\
	x&\in\left\{-1,-2+\sqrt{3},-2-\sqrt{3}\right\}
\end{align*}
Let $x_1=1$, $x_2=-2+\sqrt{3}$, $x_3=-2-\sqrt{3}$. Then:
\begin{align*}
	y_1&=1 \\
	y_2&=\frac{1+(-2+\sqrt{3})}{1+(-2+\sqrt{3})^2}
	=\frac{-1+\sqrt{3}}{1+4-4\sqrt{3}+3}
	=\frac{-1+\sqrt{3}}{8-4\sqrt{3}}
	=\frac{1}{4}\cdot\frac{-1+\sqrt{3}}{2-\sqrt{3}}\cdot\frac{2+\sqrt{3}}{2+\sqrt{3}} \\
	&=\frac{-2-\sqrt{3}+2\sqrt{3}+3}{4(4-3)}=\frac{1+\sqrt{3}}{4} \\
	y_3&=\frac{1+(-2-\sqrt{3})}{1+(-2-\sqrt{3})^2}
	=\frac{-1-\sqrt{3}}{1+4+4\sqrt{3}+3}
	=\frac{1}{4}\cdot\frac{-1-\sqrt{3}}{2+\sqrt{3}}\cdot\frac{2-\sqrt{3}}{2-\sqrt{3}} \\
	&=\frac{-2+\sqrt{3}-2\sqrt{3}+3}{4(4-3)}
	=\frac{1-\sqrt{3}}{4}
\end{align*}
Equation of a line between $(x_2,y_2)$ and $(x_3,y_3)$:
\begin{align*}
	y-\paren{\frac{1+\sqrt{3}}{4}}&=\paren{\frac{\frac{1-\sqrt{3}}{4}-\frac{1+\sqrt{3}}{4}}{-2+\sqrt{3}-(-2-\sqrt{3})}}\paren{x-\paren{-2+\sqrt{3}}} \\
	y-\frac{1+\sqrt{3}}{4}&=\paren{\frac{\frac{1-\sqrt{3}-1-\sqrt{3}}{4}}{-2+\sqrt{3}+2+\sqrt{3}}}\paren{x+2-\sqrt{3}} \\
	y-\frac{1+\sqrt{3}}{4}&=\paren{\frac{-2\sqrt{3}}{8\sqrt{3}}}\paren{x+2-\sqrt{3}} \\
	y-\frac{1+\sqrt{3}}{4}&=-\frac{1}{4}\paren{x+2-\sqrt{3}}
\end{align*}
Check if $(x_1,y_1)$ is on the line:
\begin{align*}
	1-\frac{1+\sqrt{3}}{4}&\stackrel{?}{=}-\frac{1}{4}\paren{1+2-\sqrt{3}} \\
	\frac{4-(1+\sqrt{3})}{4}&\stackrel{?}{=}-\frac{3-\sqrt{3}}{4} \\
	4-1-\sqrt{3}&\stackrel{?}{=}3-\sqrt{3} \\
	3-\sqrt{3}&=3-\sqrt{3}
\end{align*}
Therefore the three inflection points lie on a straight line.

\problem{15}


\end{document}
