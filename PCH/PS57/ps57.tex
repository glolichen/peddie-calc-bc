\documentclass{article}

\usepackage[letterpaper,portrait,top=0.4in, left=0.6in, right=0.6in, bottom=1in]{geometry}

\usepackage{amsmath, amsfonts, amsthm, amssymb}
\usepackage{graphicx, float}
\usepackage{mathtools}
\usepackage{titlesec}
\usepackage{interval}
\usepackage{hyperref}
\usepackage{siunitx}
\usepackage{titling}
\usepackage{vwcol}
\usepackage{setspace}
\usepackage{empheq}
\usepackage{cancel}
\usepackage{esdiff}
\usepackage{multicol}
\usepackage{mdframed}
\usepackage{esdiff}
\usepackage{tikzsymbols}
\usepackage{multicol}
\usepackage{tikz}
\usepackage{varwidth}

\intervalconfig {
	soft open fences
}

\newcommand{\alignedintertext}[1]{%
  \noalign{%
    \vskip\belowdisplayshortskip
    \vtop{\hsize=\linewidth#1\par
    \expandafter}%
    \expandafter\prevdepth\the\prevdepth
  }%
}

\newtheorem{lemma}{Lemma}

\renewcommand{\qedsymbol}{\Smiley[1.3]}
\newcommand*{\paren}[1]{\ensuremath\left(#1\right)}
\newcommand*{\problem}[1]{\section*{Problem #1}}
\newcommand*{\aps}{\section*{AP Corner}}
\newcommand*{\limit}[2][x]{\ensuremath{\displaystyle\lim_{#1\to#2}}}
\newcommand*{\Limit}[3][x]{\ensuremath{\displaystyle\lim_{#1\to#2}\left[#3\right]}}
\newcommand*{\deriv}[1][x]{\ensuremath{\dfrac{\mathrm{d}}{\mathrm{d}#1}}}
\newcommand*{\Deriv}[2][x]{\ensuremath{\dfrac{\mathrm{d}}{\mathrm{d}#1}\left[#2\right]}}
\newcommand*{\abs}[1]{\ensuremath{\left|#1\right|}}

\newcommand*{\eps}{\varepsilon}

\newcommand*{\floor}[1]{\ensuremath{\lfloor #1\rfloor}}

\DeclareMathOperator{\DNE}{DNE}

%opening
\title{Problem Set \#57}
\author{Jayden Li}
\date{April 3, 2024}

\allowdisplaybreaks

\begin{document}
\setstretch{1.25}
\fontsize{12pt}{12pt}\selectfont
\setlength{\abovedisplayskip}{0pt}
\maketitle

\problem{1}
\begin{align*}
	\Limit{3}{2f(x)-g(x)}&=4 \\
	2\limit{3}f(x)-\limit{3}g(x)&=4 \\
	2f(3)-g(3)&=4 \\
	10-g(3)&=4 \\
	\Aboxed{g(3)&=6}
\end{align*}

\problem{2}
\begin{align*}
	\limit{a}f(x)&=\limit{-1}\paren{x+2x^3}^4 \\
	&=\paren{\Limit{-1}{x+2x^3}}^4 \\
	&=\paren{\limit{-1}x+\limit{-1}2x^3}^4 \\
	&=\paren{-1+2\limit{-1}x^3}^4 \\
	&=\paren{-1+2\paren{\limit{-1}x}^3}^4 \\
	&=\paren{-1+2\paren{-1}^3}^4 \\
	&=f(a)
\end{align*}
Because $\limit{a}f(x)=f(a)$, $f$ is continuous at $a$ by the definition of continuity. 

\problem{3}
Let $a\in(2,\infty)$.
\begin{align*}
	\limit{a}f(x)&=\limit{a}\frac{2x+3}{x-2} \\
	&=\frac{\Limit{a}{2x+3}}{\Limit{a}{x-2}} \\
	\intertext{(this is allowed because $x-2\neq0$ since $a\neq2$.)}
	&=\frac{\limit{a}2x+\limit{a}3}{\limit{a}x-\limit{a}2} \\
	&=\frac{2a+3}{a-2} \\
	&=f(a)
\end{align*}
Because $\limit{a}f(x)=f(a)$, $f$ is continuous at $a$ by the definition of continuity for all $a\in(2,\infty)$.

\problem{4}
\begin{itemize}
	\item[(a)]
	$f$ is not defined at $1$ so the function is discontinuous at $x=1$.

	\item[(b)]
	\begin{align*}
		\limit{1^+}f(x)&=\frac{1}{1}=1 \\
		\limit{1^-}f(x)&=1-(1)^2=0
	\end{align*}
	$\limit{1}f(x)$ DNE so the function is discontinuous at $x=1$.

	\item[(c)]
	\begin{align*}
		\limit{0^+}f(x)&=1-(0^2)=1 \\
		\limit{0^-}f(x)&=\cos0=1 \\
		f(0)&=0
	\end{align*}
	$\limit{0}f(x)=1\neq0=f(0)$ so the function is discontinuous at $x=0$.
\end{itemize}

\problem{5}
\begin{itemize}
	\item[(a)]
	\begin{align*}
		x^2+5x+6&\neq0 \\
		(x+2)(x+3)&\neq0 \\
		x&\not\in\{-2,-3\}
	\end{align*}
	Domain is $(-\infty,-3)\cup(-3,-2)\cup(-2,\infty)$. $F$ is a rational function so it is continuous on its domain.

	\item[(b)]
	\begin{align*}
		2x-1&\geq0 \\
		2x&\geq1 \\
		x&\geq\frac{1}{2}
	\end{align*}
	Domain is $\interval[scaled,open right]{\dfrac{1}{2}}{\infty}$. $x^2$ is a polynomial and continuous. $2x$ is a polynomial and continuous. $1$ is a constant and continuous. $2x-1$ is the difference of two continuous function and is continuous. The root function is continuous on all points on its domain so it is continuous on the image of $2x-1$. $x^2+\sqrt{2x-1}$ is the sum of two continuous functions and is continuous on all points on its domain.

	\item[(c)]
	Domain is $\mathbb{R}$. Cosine is continuous on $\mathbb{R}$ and the image of $1-x^2$ is a subset of $\mathbb{R}$.

	\item[(d)]
	Domain is $(0,\infty]$. $\sqrt{x}$ is a root function and continuous on its domain. Sine is continuous on $\mathbb{R}$. $F$ is the product of two continuous function and is therefore continuous. 
\end{itemize}

\problem{6}
This function is continuous on its domain. Let $f(x)=1/x$ and $g(x)=1+\sin x$, then $y=f(g(x))$. $f$ is a rational function and continuous on its domain, which is $x\neq0$. $g$ is continuous on $\mathbb{R}$. So the graph is continuous on all $g(x)\neq0$.
\begin{align*}
	1+\sin x&\neq0 \\
	\sin x&\neq-1 \\
	x&\neq-\frac{\pi}{2}+\pi n
\end{align*}
Discontinuities are at all points $-\pi/2+\pi n$ where $n\in\mathbb{Z}$.

\problem{7}
\begin{multicols*}{2}
	\begin{itemize}
		\item[(a)]
		\begin{align*}
			&\limit{4}\frac{5+\sqrt{x}}{\sqrt{5+x}}
			\intertext{Denominator not equal to $0$: $\sqrt{5+4}=3\neq0$}
			={}&\frac{\Limit{4}{5+\sqrt{x}}}{\limit{4}\sqrt{5+x}} \\
			={}&\frac{\limit{4}5+\sqrt{\limit{4}x}}{\sqrt{\limit{4}5+\limit{4}x}} \\
			\intertext{$x$ is continuous.}
			={}&\frac{5+2}{\sqrt{5+4}} \\
			={}&\boxed{\frac{7}{9}}
		\end{align*}
		\columnbreak
		\item[(b)]
		\begin{align*}
			&\limit{\pi/4}x\cos^2 x \\
			={}&\limit{\pi/4}x\cdot\paren{\limit{\pi/4}\cos x}^2
			\intertext{It is known that cosine and $x$ are continuous.}
			={}&\frac{\pi}{4}\paren{\frac{\sqrt{2}}{2}}^2 \\
			={}&\boxed{\frac{\pi}{8}}
		\end{align*}
	\end{itemize}
\end{multicols*}

\problem{8}
$x^2$ is a polynomial and continuous on $\mathbb{R}$. $\sqrt{x}$ is a root function and continu

\problem{9}

\problem{10}

\problem{11}

\problem{12}

\problem{13}

\problem{14}

\end{document}
