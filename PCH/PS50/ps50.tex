\documentclass{article}

\usepackage[letterpaper,portrait,top=0.4in, left=0.6in, right=0.6in, bottom=1in]{geometry}

\usepackage{amsmath, amsfonts, amsthm, amssymb}
\usepackage{graphicx, float}
\usepackage{mathtools}
\usepackage{titlesec}
\usepackage{interval}
\usepackage{hyperref}
\usepackage{siunitx}
\usepackage{titling}
\usepackage{vwcol}
\usepackage{setspace}
\usepackage{empheq}
\usepackage{cancel}
\usepackage{esdiff}
\usepackage{multicol}
\usepackage{mdframed}
\usepackage{esdiff}
\usepackage{multicol}
\usepackage{tikz}
\usepackage{varwidth}

\intervalconfig {
	soft open fences
}

\newcommand{\alignedintertext}[1]{%
	\noalign{%
		\vskip\belowdisplayshortskip
		\vtop{\hsize=\linewidth#1\par
		\expandafter}%
		\expandafter\prevdepth\the\prevdepth
	}%
}

% \allowdisplaybreaks

%opening
\title{Problem Set \#50}
\author{Jayden Li}
\date{February 8, 2024}

\allowdisplaybreaks

\begin{document}
\setlength{\abovedisplayskip}{0pt}
\fontsize{12pt}{12pt}\selectfont
\maketitle

\section*{Problem 9}
\begin{itemize}
	\begin{minipage}[t]{0.5\linewidth}
		\item[(a)]
		I think Cartesian is easier.
		
		Cartesian:
		$\begin{aligned}[t]
			\Aboxed{y&=\frac{\pi}{6}x}
		\end{aligned} \\$

		Polar:
		$\begin{aligned}[t]
			r\sin\theta&=\frac{\pi}{6}(r\cos\theta) \\
			\tan\theta&=\frac{\pi}{6} \\
			\Aboxed{\theta&=\arctan\frac{\pi}{6}+\pi n}
		\end{aligned}$
	\end{minipage}
	\begin{minipage}[t]{0.5\linewidth}
		\item[(b)]
		I think Cartesian is easier.

		Cartesian:
		$\begin{aligned}[t]
			\Aboxed{x&=3}
		\end{aligned} \\$

		Polar:
		$\begin{aligned}[t]
			r\cos\theta&=3 \\
			\Aboxed{r&=3\sec\theta}
		\end{aligned}$
	\end{minipage}

\end{itemize}

\section*{Problem 10}
\begin{align*}
	r_1=3+\cos\theta&=2=r^2 \\
	\cos\theta&=-1 \\
	\theta&\in\left\{\frac{2\pi}{3},\frac{4\pi}{3}\right\}
\end{align*}
Let $A$ be the point with $\theta=\dfrac{2\pi}{3}$ and $B$ be the point with $\theta=\dfrac{4\pi}{3}$. Any line from $A$ to the origin must satisfy $\theta=\dfrac{2\pi}{3}$ or $\theta=\dfrac{2\pi}{3}+\pi$ (because a line rotated $\ang{180}$ is the same line). When $\theta=\dfrac{2\pi}{3}$, $r_1=3+2\cos\dfrac{2\pi}{3}=3+(-1)=2$, which is $A$. When $\theta=\dfrac{2\pi}{3}+\pi$, $r_1=3+2\cos\left(\dfrac{5\pi}{3}\right)=3+1=4$.

Therefore $C$ has rectangular coordinates $(r_1\cos\theta,r_2\sin\theta)=\left(4\cos\dfrac{5\pi}{3},4\sin\dfrac{5\pi}{3}\right)=(2,-2\sqrt{3})$.

$B$ has polar coordinates $\left(2,\dfrac{4\pi}{3}\right)$ and rectangular coordinates $\left(2\cos\dfrac{4\pi}{3},2\sin\dfrac{4\pi}{3}\right)=(-1,-\sqrt{3})$.

The equation of $BC$ is:
\begin{align*}
	y-y_0&=\frac{y_1-y_0}{x_1-x_0}(x-x_0) \\
	y+2\sqrt{3}&=\frac{-2\sqrt{3}+\sqrt{3}}{2+1}(x-2) \\
	y&=-\frac{\sqrt{3}}{3}x+\frac{2\sqrt{3}}{3}-\frac{6\sqrt{3}}{3} \\
	y&=-\frac{\sqrt{3}}{3}x-\frac{4\sqrt{3}}{3}
\end{align*}

Substituting into $r_2$:
\begin{align*}
	r_2&=2 \\
	x^2+y^2&=r^2 \\
	x^2+\left(-\frac{\sqrt{3}}{3}x-\frac{4\sqrt{3}}{3}\right)^2&=4 \\
	x^2+\frac{3}{9}x^2+2\left(\frac{\sqrt{3}}{3}x\right)\left(\frac{4\sqrt{3}}{3}\right)+\frac{48}{9}-4&=0 \\
	x^2+\frac{1}{3}x^2+\frac{8}{3}x+\frac{4}{3}&=0 \\
	4x^2+8x+4&=0 \\
	x^2+2x+1&=0 \\
	(x+1)^2&=0 \\
	x&=-1
\end{align*}
There is only one intersection between the line $BC$ and the circle given by $r_2$, and the intersection has $x=-1$, which is the $x$-coordinate of $B$. Therefore $BC$ is a tangent line of $r_2$ at point $B$.

\qed

\end{document}