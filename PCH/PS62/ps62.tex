\documentclass{article}

\usepackage[letterpaper,portrait,top=0.4in, left=0.6in, right=0.6in, bottom=1in]{geometry}

\usepackage{amsmath, amsfonts, amsthm, amssymb}
\usepackage{graphicx, float}
\usepackage{mathtools}
\usepackage{titlesec}
\usepackage{interval}
\usepackage{hyperref}
\usepackage{siunitx}
\usepackage{titling}
\usepackage{vwcol}
\usepackage{setspace}
\usepackage{empheq}
\usepackage{cancel}
\usepackage{esdiff}
\usepackage{multicol}
\usepackage{mdframed}
\usepackage{esdiff}
\usepackage{tikzsymbols}
\usepackage{multicol}
\usepackage{tikz}
\usepackage{varwidth}

\intervalconfig {
	soft open fences
}

\newcommand{\alignedintertext}[1]{%
  \noalign{%
    \vskip\belowdisplayshortskip
    \vtop{\hsize=\linewidth#1\par
    \expandafter}%
    \expandafter\prevdepth\the\prevdepth
  }%
}

\newtheorem{lemma}{Lemma}

\renewcommand{\qedsymbol}{\Smiley[1.3]}
\newcommand*{\paren}[1]{\ensuremath\left(#1\right)}
\newcommand*{\problem}[1]{\section*{Problem #1}}
\newcommand*{\aps}{\section*{AP Corner}}
\newcommand*{\limit}[2][x]{\ensuremath{\displaystyle\lim_{#1\to#2}}}
\newcommand*{\Limit}[3][x]{\ensuremath{\displaystyle\lim_{#1\to#2}\left[#3\right]}}
\newcommand*{\deriv}[1][x]{\ensuremath{\dfrac{\mathrm{d}}{\mathrm{d}#1}}}
\newcommand*{\Deriv}[2][x]{\ensuremath{\dfrac{\mathrm{d}}{\mathrm{d}#1}\left[#2\right]}}
\newcommand*{\iinteg}[2][x]{\ensuremath{\displaystyle\int #2\;\mathrm{d}#1}}
\newcommand*{\dinteg}[4][x]{\ensuremath{\displaystyle\int_{#2}^{#3}#4\;\mathrm{d}#1}}
\newcommand*{\abs}[1]{\ensuremath{\left|#1\right|}}
\newcommand*{\eps}{\varepsilon}
\newcommand*{\floor}[1]{\ensuremath{\lfloor #1\rfloor}}
\newcommand*{\cbrt}[1]{\ensuremath{\sqrt[3]{#1}}}

\DeclareMathOperator{\DNE}{DNE}

%opening
\title{Problem Set \#62}
\author{Jayden Li}
\date{April 9, 2024}

\allowdisplaybreaks

\begin{document}
\setstretch{1.25}
\fontsize{12pt}{12pt}\selectfont
\setlength{\abovedisplayskip}{0pt}
\maketitle

\problem{1}
The absolute minimum is the value such that it is greater than all values on an interval. The local minimum is a value such that it is greater than all points in its neighborhood.

\problem{2}
There is no absolute minimum. \newline
Absolute maximum: $y=3$. \newline
Local minima: $y=1,y=1$. \newline
Local maximum: $y=3$.

\problem{3}
\begin{center}
	\includegraphics*[width=0.6\linewidth]{q3.png}

	\textit{(This is, in fact, problem 3; sorry about putting 4 on the picture!)}
\end{center}

\problem{6}
\begin{itemize}
	\item[(b)]
	\begin{align*}
		f'(x)=5x^4+1&=0 \\
		x^4&=-\frac{1}{5}
	\end{align*}
	No real solution for $x$, so no CPs. Endpoints: $x=-1,x=1$.
	\begin{center}
		$\begin{array}{c|c}
			x & f(x) \\
			\hline
			-1 & -1 \\
			1 & 3
		\end{array}$
	\end{center}
	Absolute min: $y=-1$, absolute max: $y=3$.

	\item[(c)]
	\begin{align*}
		f'(x)=12x^3-18x^2+12x&=0 \\
		6x\paren{2x^2-3x+2}&=0 \\
		x=0\text{ or }x&=\cancel{\frac{3\pm\sqrt{9-16}}{4}}
	\end{align*}
	$0\in\left[-\frac{1}{2},\frac{1}{2}\right]$.

	CPs: $x=0$. Endpoints: $x=-1/2,x=1/2$.
	\begin{center}
		$\begin{array}{c|c}
			x & f(x) \\
			\hline
			-1/2 & 2.44 \\
			0 & 0 \\
			1/2 & 0.94
		\end{array}$
	\end{center}
	Absolute min: $y=0$, absolute max: $y=2.44$.

	\item[(d)]
	\begin{align*}
		f'(x)=\frac{\Deriv{1}\paren{x^5+x+1}-1\cdot\Deriv{x^5+x+1}}{\paren{x^5+x+1}^2}&=0 \\
		\frac{5x^4+1}{\paren{x^5+x+1}^2}&=0 \\
		5x^4+1&=0 \\
		x^4&=-\frac{1}{5}
	\end{align*}
	No CPs. Endpoints: $x=-1/2,x=1$.
	\begin{center}
		$\begin{array}{c|c}
			x & f(x) \\
			\hline
			-1/2 & 2.13 \\
			1 & 0.33
		\end{array}$
	\end{center}
	Absolute min: $y=0.33$, absolute max: $y=2.13$.

	\item[(e)]
	\begin{align*}
		f'(x)=\frac{\Deriv{x+1}\paren{x^2+1}-\Deriv{x^2+1}\paren{x+1}}{\paren{x^2+1}^2}&=0 \\
		\frac{x^2+1-2x(x+1)}{\paren{x^2+1}^2}&=0 \\
		-x^2-2x+1&=0 \\
		x^2+2x-1&=0 \\
		x&=\frac{-2\pm\sqrt{4+4}}{2} \\
		x&=\frac{-2}{2}\pm\frac{2\sqrt{2}}{2} \\
		x&=-1\pm\sqrt{2}
	\end{align*}
	$-1+\sqrt{2}\approx0.41\in\left[-1,\frac{1}{2}\right],-1-\sqrt{2}\approx-2.41\not\in\left[-1,\frac{1}{2}\right]$.
	
	CPs: $x=-1+\sqrt{2}$. Endpoints: $x=-1,x=1/2$.
	\begin{center}
		$\begin{array}{c|c}
			x & f(x) \\
			\hline
			-1+\sqrt{2} & 1.21 \\
			-1 & 0 \\
			1/2 & 1.2
		\end{array}$
	\end{center}
	Absolute min: $y=0$, absolute max: $y=1.21$.

	\item[(f)]
	\begin{align*}
		f'(x)=\frac{\Deriv{x}\paren{x^2-1}-\Deriv{x^2-1}\paren{x}}{\paren{x^2-1}^2}&=0 \\
		\frac{x^2-1-2x^2}{\paren{x^2-1}^2}&=0 \\
		-x^2-1&=0 \\
		x^2&=-1 \\
	\end{align*}	
	No CPs. Endpoints: $x=0,x=5$.
	\begin{center}
		$\begin{array}{c|c}
			x & f(x) \\
			\hline
			0 & 0 \\
			5 & 0.21
		\end{array}$
	\end{center}
	Absolute min: $y=0$, absolute max: $y=0.21$.
\end{itemize}

\problem{7}
\begin{center}
	\includegraphics*[width=0.6\linewidth]{q7.png}
\end{center}
\begin{align*}
	f'(x)&=0 \\
	\Deriv{x+\frac{3}{x^2}}&=0 \\
	1+3\cdot\Deriv{x^{-2}}&=0 \\
	f'(x)=1-\frac{6}{x^3}&=0 \\
	x^3-6&=0 \\
	x&=\cbrt{6}
\end{align*}
Just to be sure, we can calculate the second derivative to make sure it really is a local minimum.
\begin{align*}
	f''(x)&=\Deriv{1-\frac{6}{x^3}} \\
	&=6\cdot\Deriv{x^{-3}} \\
	&=-\frac{18}{x^4} \\
	f''(\cbrt{6})&=-\frac{18}{6\cbrt{6}}
\end{align*}
Which is obviously negative, hence we have our only local minimum $x=\cbrt{6}$.

\problem{8}
\begin{equation*}
	f(x)=x^a(1-x)^b
\end{equation*}
\begin{align*}
	f'(x)=ax^{a-1}\cdot(1-x)^b+x^a\cdot b(1-x)^{b-1}\cdot(-1)&=0 \\
	ax^{a-1}(1-x)^b-bx^a(1-x)^{b-1}&=0 \\
	ax^{a-1}(1-x)^{b-1}\cdot(1-x)&=bx^{a-1}(1-x)^{b-1}\cdot x \\
	a(1-x)&=bx \\
	a-ax&=bx \\
	x(a+b)&=a \\
	x&=\frac{a}{a+b}=\frac{a}{a\paren{1+\frac{b}{a}}}=\frac{1}{1+\frac{b}{a}}
\end{align*}
$\dfrac{b}{a}$ is always positive since $a$ and $b$ are positive, so the denominator is greater than the numerator. Hence $\dfrac{a}{a+b}\in[0,1]$ and is a CP. Endpoints: $x=0,x=1$.
\begin{center}
	$\begin{array}{c|c}
		x & f(x) \\
		\hline
		0 & 0^a\cdot(1-0)^b=0 \\
		1 & 1^a(1-1)^b=0 \\
		\dfrac{a}{a+b} & \paren{\dfrac{a}{a+b}}^a\paren{1-\dfrac{a}{a+b}}^b 
	\end{array}$
\end{center}
Because $\dfrac{a}{a+b}\in[0,1]$, $\paren{1-\dfrac{a}{a+b}}\in[0,1]$, and since $a$ and $b$ are also positive, $\paren{\dfrac{a}{a+b}}^a$ and $\paren{1-\dfrac{a}{a+b}}^b$ must be positive and greater than $0$. Therefore, the maximum value of $f$ on $[0,1]$ is:
\begin{equation*}
	\paren{\dfrac{a}{a+b}}^a\paren{1-\dfrac{a}{a+b}}^b=\paren{\dfrac{a}{a+b}}^a\paren{\frac{a+b-a}{a+b}}^b=\boxed{\paren{\dfrac{a}{a+b}}^a\paren{\frac{b}{a+b}}^b}
\end{equation*}

\problem{9}
\begin{equation*}
	f(x)=x^{101}+x^{51}+x+1
\end{equation*}
\begin{align*}
	f'(x)=101x^{100}+51x^{50}+1&=0 \\
	101\paren{x^50}^2+51x^{50}+1&=0 \\
	x^{50}&=\frac{-51\pm\sqrt{51^2-4\cdot1\cdot101}}{101\cdot2} \\
	x^{50}&=\frac{-51\pm\sqrt{2197}}{202} \\
	x&=\sqrt[50]{\frac{-51\pm\sqrt{2197}}{202}} \\
\end{align*}
$\sqrt{2197}\approx46.87$, so RHS is always negative. The even root of a negative number is not real, so there are no values $x$ such that $f'(x)=0$. Thus, there are no critical points.	 

\problem{10}
\begin{itemize}
	\item[(a)] I'm going to guess that all rational numbers are local maxima, and all irrational numbers are local minima. There is will be an irrational number in any open interval containing any rational number (density of irrational numbers), and $1/q$ will definitely be larger than $0$, since $q$ is positive. Likewise, there must be a rational number in the neighborhood around any irrational number, and the value of $f$ at that irrational number ($0$) will definitely be less than any value $1/q$ at the rational number. \textit{What about rational numbers on that open interval???}
	\item[(b)] $1/n$ for all $n\in\mathbb{N}_1$ are local maxima. Irrational numbers and rational numbers that cannot be expressed in that form are dense around $1/n$ and $1/n$ is definitely not dense so any neighborhood around $1/n$ must contain only $0$s. Likewise any number that cannot be expressed as $1/n$ must be local minima since $0$ is the minimum possible value of $g$.
\end{itemize}
\textit{My answers/reasoning here are probably wrong; I will ask about these tommorrow at conference.}

\problem{11}
\begin{itemize}
	\item[(a)]
	\begin{align*}
		v'(r)&=0 \\
		\Deriv[r]{kr_0r^2-kr^3}&=0 \\
		2kr_0r-3kr^2&=0 \\
		\intertext{$r=0$ is a solution, but $0\not\in[r_0/2,r_0]$ so we won't be considering it.}
		2r_0-3r&=0 \\
		r&=\frac{2r_0}{3}
	\end{align*}
	CPs: $r=2r_0/3$. Endpoints: $r=r_0/2,r_0$.
	\begin{center}
		$\begin{array}{c|c}
			r & v(r) \\
			\hline
			r_0/2 & k\paren{\frac{2r_0}{2}-\frac{r_0}{2}}\paren{\frac{r_0}{2}}^2=k\cdot\frac{r_0}{2}\cdot\frac{r_0^2}{4}=\frac{kr_0^3}{8} \\
			2r_0/3 & k\paren{\frac{3r_0}{3}-\frac{2r_0}{3}}\paren{\frac{2r_0}{3}}^2=k\cdot\frac{r_0}{3}\cdot\frac{4r_0^2}{9}=\frac{4kr_0^3}{27} \\
			r_0 & k(r_0-r_0)(r_0)^2=0
		\end{array}$
	\end{center}
	Because $4/27>1/8$ and both $k$ and $r_0$ are positive, $v$ has an absolute maximum at $r=2r_0/3$. This is the same as the experimental evidence, as the velocity of the airstream is th highest at $r=2r_0/3$.

	\item[(b)]
	$v(r)=\dfrac{4kr_0^3}{27}$.

	\item[(c)]
	\phantom{}
	
	\begin{center}
		\includegraphics*[width=0.6\linewidth]{q11c.png}

		\textit{(I seem to be quite bad at labeling questions, this one is also supposed to be 11c, not 11b.)}
	\end{center}
\end{itemize}

\end{document}
