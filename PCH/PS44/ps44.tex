\documentclass{article}

\usepackage[letterpaper,portrait,top=0.4in, left=0.6in, right=0.6in, bottom=1in]{geometry}

\usepackage{amsmath, amsfonts, amsthm, amssymb}
\usepackage{graphicx, float}
\usepackage{mathtools}
\usepackage{titlesec}
\usepackage{interval}
\usepackage{hyperref}
\usepackage{titling}
\usepackage{vwcol}
\usepackage{setspace}
\usepackage{empheq}
\usepackage{cancel}
\usepackage{esdiff}

\intervalconfig {
	soft open fences
}

\newcommand{\alignedintertext}[1]{%
	\noalign{%
		\vskip\belowdisplayshortskip
		\vtop{\hsize=\linewidth#1\par
		\expandafter}%
		\expandafter\prevdepth\the\prevdepth
	}%
}

% \allowdisplaybreaks

%opening
\title{Problem Set \#44}
\author{Jayden Li}
\date{January 29, 2024}

\begin{document}

\fontsize{12pt}{12pt}\selectfont

\maketitle

\section*{Problem 1}
\begin{itemize}
\item[(a)]
	\begin{equation*}
		\diff{}{x}\left[x^2+f(x)\right]=\boxed{2x+f'(x)}
	\end{equation*}

\item[(b)]
	\begin{equation*}
		\diff{}{x}\left[x^2f(x)\right]=\diff{}{x}\left[x^2\right]f(x)+x^2\diff{}{x}\left[f(x)\right]=\boxed{2xf(x)+x^2f'(x)}
	\end{equation*}

\item[(c)]
	\begin{equation*}
		\diff{}{x}\left[c+x+f(x)\right]=\cancel{\diff{}{x}c}+\diff{}{x}x+\diff{}{x}f(x)=\boxed{1+f'(x)}
	\end{equation*}

\item[(d)]
	\begin{equation*}
		\diff{}{x}\left[f(x^2)\right]=f'(x^2)\cdot\diff{}{x}\left[x^2\right]=\boxed{2xf'(x^2)}
	\end{equation*}

\item[(e)]
	\begin{align*}
		\diff{}{x}\left[xf(x)+f(cx)+cf(x)\right]&=\diff{}{x}\left[xf(x)\right]+\diff{}{x}\left[f(cx)\right]+\diff{}{x}\left[cf(x)\right] \\
		&=\diff{}{x}\left[x\right]f(x)+x\diff{}{x}\left[f(x)\right]+f'(cx)\diff{}{x}\left[cx\right]+c\diff{}{x}\left[f(x)\right] \\
		&=\boxed{f(x)+xf'(x)+cf'(cx)+cf'(x)}
	\end{align*}
\end{itemize}


\section*{Problem 2}
\begin{itemize}
\item[(a)]
	\begin{align*}
		x^2+y^2&=1 \\
		x^2+(f(x))^2&=1 \\
		\diff{}{x}\left[x^2+(f(x))^2\right]&=\diff{}{x}1 \\
		2x+2f(x)f'(x)&=0 \\
		f'(x)&=\frac{-2x}{2f(x)} \\
		\Aboxed{f'(x)&=-\frac{x}{f(x)}}
	\end{align*}
	The tangent line to the circle defined by $x^2+y^2=1$ at the point $(x,y)$ has slope $-\dfrac{x}{y}$.

\item[(b)]
	Let $y=f(x)$. The equation $x^2+y^2=1$ is equal to $x^2+(f(x))^2=1$. We have shown in (a) that $f'(x)=-\dfrac{x}{f(x)}$, and substituting $f(x)$ for $y$ gives \boxed{\diff{y}{x}=-\dfrac{x}{y}}.

	The derivative at $\left(\dfrac{\sqrt{2}}{2},\dfrac{\sqrt{2}}{2}\right)$ is $-\dfrac{\sqrt{2}/2}{\sqrt{2}/2}=\boxed{1}$
	
\item[(c)]
	\begin{equation*}
		\boxed{y-\frac{\sqrt{2}}{2}=-\left(x-\frac{\sqrt{2}}{2}\right)}
	\end{equation*}

\item[(d)]
	The center of the circle is $O(0,0)$. The slope of the radius $OP$ is`' $\dfrac{y-0}{x-0}=\dfrac{y}{x}$. The slope of any line perpendicular to $OP$ is $-\dfrac{x}{y}$. The tangent line of the circle at $(x,y)$ is $-\dfrac{x}{y}$. \qed
\end{itemize}


\section*{Problem 3}
\begin{itemize}
\item[(a)]
	Because $y$ is not a function as it fails the vertical line test.

\item[(b)]
	\setlength{\abovedisplayskip}{0pt}
	\begin{align*}
		x&=y^5-5y^3+4y \\
		\diff{}{x}x&=\diff{}{x}\left[y^5-5y^3+4y\right] \\
		1&=5y^4\diff{y}{x}-15y^2\diff{y}{x}+4\diff{y}{x} \\
		1&=\diff{y}{x}\left(5y^4-15y^2+4\right) \\
		\Aboxed{\diff{y}{x}&=\frac{1}{5y^4-15y^2+4}}
	\end{align*}

\item[(c)]
	\setlength{\abovedisplayskip}{0pt}
	\begin{align*}
		y-y_0&=\frac{1}{5y_0^4-15y_0^2+4}(x-x_0) \\
		y-1&=\frac{1}{5(1)^4-15(1)^2+4}(x-0) \\
		\Aboxed{y-1&=-\frac{1}{6}x} \\
	\end{align*}
\end{itemize}


\section*{Problem 4}
\begin{itemize}
\item[(a)]
	\begin{align*}
		x^2y^2+x\sin y&=4 \\
		\diff{}{x}\left[x^2y^2+x\sin y\right]&=\diff{}{x}4 \\
		\diff{}{x}\left[x^2\right]y^2+x^2\diff{}{x}\left[y^2\right]+\diff{}{x}\left[x\right]\sin y+x\diff{}{x}\left[\sin y\right]&=0 \\
		2xy^2+2x^2y\diff{y}{x}+\sin y+x\cos(y)\diff{y}{x}&=0 \\
		\diff{y}{x}\left(2x^2y+x\cos y\right)&=-2xy^2-\sin y \\
		\Aboxed{\diff{y}{x}&=-\frac{2xy^2+\sin y}{2x^2y+x\cos y}}
	\end{align*}

\item[(b)]
	\begin{align*}
		4\cos x\sin y&=1 \\
		\diff{}{x}\left[4\cos x\sin y\right]&=\diff{}{x}1 \\
		4\left(\diff{}{x}\left[\cos(x)\right]\sin(y)+\cos(x)\diff{}{x}\left[\sin(y)\right]\right)&=0 \\
		-4\sin x\sin y+4\cos x\cos y\diff{y}{x}&=0 \\
		4\cos x\cos y\diff{y}{x}&=4\sin x\sin y \\
		\diff{y}{x}&=\frac{4\sin x\sin y}{4\cos x\cos y} \\
		\Aboxed{\diff{y}{x}&=\tan x\tan y}
	\end{align*}

\item[(c)]
	\begin{align*}
		x\ln y+y^3&=3\ln x \\
		\diff{}{x}\left[x\ln y+y^3\right]&=\diff{}{x}\left[3\ln x\right] \\
		\ln y+x\left(\frac{1}{y}\right)\diff{y}{x}+3y^2\diff{y}{x}&=\frac{3}{x} \\
		\diff{y}{x}\left(\frac{x}{y}+3y^2\right)&=\frac{3}{x}-\ln y \\
		\diff{y}{x}\left(x^2+3xy^3\right)&=3y-xy\ln y \\
		\Aboxed{\diff{y}{x}&=\frac{3y-xy\ln y}{x^2+3xy^3}}
	\end{align*}

\item[(d)]
	\begin{align*}
		\tan(x-y)&=\frac{y}{1+x^2} \\
		\diff{}{x}\left[\tan(x-y)\right]&=\diff{}{x}\left[\frac{y}{1+x^2}\right] \\
		\sec^2(x-y)\diff{}{x}\left[x-y\right]&=\frac{\diff{}{x}[y]\left(1+x^2\right)-y\diff{}{x}\left[1+x^2\right]}{\left(1+x^2\right)^2} \\
		\sec^2(x-y)\left(1-\diff{y}{x}\right)&=\frac{\diff{y}{x}\left(1+x^2\right)-2xy}{1+2x^2+x^4} \\
		\left(\sec^2(x-y)-\sec^2(x-y)\diff{y}{x}\right)\left(1+2x^2+x^4\right)&=\diff{y}{x}\left(1+x^2\right)-2xy \\
		\sec^2(x-y)\left(1+2x^2+x^4\right)-\diff{y}{x}\sec^2(x-y)\left(1+2x^2+x^4\right)&=\diff{y}{x}\left(1+x^2\right)-2xy \\
		\sec^2(x-y)\left(1+2x^2+x^4\right)+2xy&=\diff{y}{x}\left(1+x^2\right)+\diff{y}{x}\sec^2(x-y)\left(1+2x^2+x^4\right) \\
		\sec^2(x-y)\left(1+2x^2+x^4\right)+2xy&=\diff{y}{x}\left(1+x^2+\sec^2(x-y)\left(1+2x^2+x^4\right)\right) \\
		\Aboxed{\diff{y}{x}&=\frac{\sec^2(x-y)\left(1+2x^2+x^4\right)+2xy}{1+x^2+\sec^2(x-y)\left(1+2x^2+x^4\right)}}
	\end{align*}
\end{itemize}


\section*{Problem 5}
\begin{align*}
	\sqrt{x}+\sqrt{y}&=\sqrt{c} \\
	\diff{}{x}\left[\sqrt{x}+\sqrt{y}\right]&=\diff{}{x}\sqrt{c} \\
	\frac{1}{2\sqrt{x}}+\frac{1}{2\sqrt{y}}\diff{y}{x}&=0 \\
	\diff{y}{x}&=\frac{-\frac{1}{2\sqrt{x}}}{\frac{1}{2\sqrt{y}}} \\
	\diff{y}{x}&=-\frac{\sqrt{y}}{\sqrt{x}}
\end{align*}
The tangent line at any point $(x_0,y_0)$ on the curve is $
y-y_0=-\dfrac{\sqrt{y_0}}{\sqrt{x_0}}(x-x_0)$.

\begin{minipage}[t]{0.42\linewidth}
\begin{align*}
	\text{$x$-intercepts: }0-y_0&=-\frac{\sqrt{y_0}}{\sqrt{x_0}}(x-x_0) \\
	-y_0&=-\frac{x\sqrt{y_0}}{\sqrt{x_0}}+\frac{x_0\sqrt{y_0}}{\sqrt{x_0}} \\
	\frac{x\sqrt{y_0}}{\sqrt{x_0}}&=y_0+\sqrt{\frac{x_0^{\cancel{2}}y_0}{\cancel{x_0}}} \\
	x&=\frac{\sqrt{x_0}}{\sqrt{y_0}}\left(y_0+\sqrt{x_0y_0}\right) \\
	x&=\frac{y_0\sqrt{x_0}}{\sqrt{y_0}}+\frac{\sqrt{x_0y_0}\sqrt{x_0}}{\sqrt{y_0}} \\
	x&=\sqrt{\frac{x_0y_0^2}{y_0}}+\frac{x_0\cancel{\sqrt{y_0}}}{\cancel{\sqrt{y_0}}} \\
	x&=\sqrt{x_0y_0}+x_0
\end{align*}
\end{minipage}
\begin{minipage}[t]{0.55\linewidth}
\begin{align*}
	\text{$y$-intercepts: }y-y_0&=-\frac{\sqrt{y_0}}{\sqrt{x_0}}(0-x_0) \\
	y-y_0&=\sqrt{\frac{y_0}{x_0}}\sqrt{x_0^2} \\
	y&=\sqrt{x_0y_0}+y_0
\end{align*}
\begin{align*}
	\text{Sum of intercepts: }&\sqrt{x_0y_0}+x_0+\sqrt{x_0y_0}+y_0 \\
	=\,&\bigl(\sqrt{x_0}\bigr)^2+2\sqrt{x_0y_0}+\bigl(\sqrt{y_0}\bigr)^2 \\
	=\,&\left(\sqrt{x_0}+\sqrt{y_0}\right)^2 \\
	=\,&\bigl(\sqrt{c}\bigr)^2 \\
	=\,&c
\end{align*}
\qed
\end{minipage}


\section*{Problem 6}
\centering
\begin{minipage}[t]{0.45\linewidth}
\begin{align*}
	x^2y^2+xy&=2 \\
	\diff{}{x}\left[x^2y^2+xy\right]&=\diff{}{x}2 \\
	2xy^2+2x^2y\diff{y}{x}+y+x\diff{y}{x}&=0 \\
	\diff{y}{x}\left(2x^2y+x\right)&=-\left(2xy^2+y\right) \\
	\diff{y}{x}&=-\frac{2xy^2+y}{2x^2y+x}
\end{align*}
\end{minipage}
\begin{minipage}[t]{0.4\linewidth}
\begin{align*}
	-\frac{2xy^2+y}{2x^2y+x}&=-1 \\
	2xy^2+y&=2x^2y+x \\
	2xy^2-2x^2y&=x-y \\
	2xy(y-x)+(y-x)&=0\\
	(2xy+1)(y-x)&=0
\end{align*}
\end{minipage}
\vspace*{8pt}
\par\noindent\rule{\linewidth}{0.5pt}
\begin{equation*}
	x^2y^2+xy=2
\end{equation*}
\begin{minipage}[t]{0.4\linewidth}
	\setlength{\abovedisplayskip}{0pt}
	\begin{align*}
		2xy+1&=0 \\
		y&=-\frac{1}{2x} \\
		x^2\left(-\frac{1}{2x}\right)^2+x\left(-\frac{1}{2x}\right)&=2 \\
		\frac{x^2}{4x^2}-\frac{x}{2x}&=2 \\
		\frac{1}{4}-\frac{2}{4}&=2
	\end{align*}
	(This case is impossible.)
\end{minipage}
\begin{minipage}[t]{0.4\linewidth}
	\setlength{\abovedisplayskip}{0pt}
	\begin{align*}
		y-x&=0 \\
		y&=x \\
		x^2x^2+xx&=2 \\
		x^4+x^2-2&=0 \\
		(x^2+2)(x^2-1)&=0
	\end{align*}
	\centering
	$x=1,y=1$ or $x=-1,y=-1$
	\linebreak
	\linebreak
	\boxed{(1,1),(-1,-1)}
\end{minipage}
\flushleft


\section*{Problem 7}
If the point $(-5,0)$ is on the edge of the shadow, then the line drawn from $(-5,0)$ to the lamp must be a tangent line of the ellipse $x^2+4y^2=5$ and must intercept it at one point only.
% \linebreak
% \linebreak
Let $k$ be the $y$-coordinate of the lamp. Then the coordinates of the lamp is $(3,k)$.

\centering
\begin{minipage}[t]{0.43\linewidth}
\begin{align*}
	y-0&=\frac{k-0}{3-(-5)}(x-(-5)) \\
	y&=\frac{k}{8}(x+5) \\
	y&=\frac{kx+5k}{8}
\end{align*}
\begin{align*}
	x^2+4y^2&=5 \\
	x^2+4\left(\frac{kx+5k}{8}\right)^2&=5 \\
	x^2+4\left(\frac{k^2x^2+10k^2x+25k^2}{64}\right)&=5 \\
	x^2+\frac{k^2x^2}{16}+\frac{5k^2x}{8}+\frac{25k^2}{16}&=5 \\
	\left(1+\frac{k^2}{16}\right)x^2+\frac{5k^2x}{8}+\left(\frac{25k^2}{16}-5\right)&=0 \\
	\frac{-\frac{5k^2}{8}\pm\sqrt{\left(\frac{5k^2}{8}\right)^2-4\left(1+\frac{k^2}{16}\right)\left(\frac{25k^2}{16}-5\right)}}{2\left(1+\frac{k^2}{16}\right)}&=x
\end{align*}
\end{minipage}
\hfill
\begin{minipage}[t]{0.52\linewidth}
\vspace*{10pt}
Because there is only one intercept, there must only be one solution for $x$, and the discriminant must be 0.
\begin{align*}
	\left(\frac{5k^2}{8}\right)^2-4\left(1+\frac{k^2}{16}\right)\left(\frac{25k^2}{16}-5\right)&=0 \\
	\frac{25k^4}{64}-4\left(\frac{25k^2}{16}-5+\frac{25k^4}{256}-\frac{5k^2}{16}\right)&=0 \\
	\cancel{\frac{25k^4}{64}}-\frac{25k^2}{4}+20-\cancel{\frac{25k^4}{64}}+\frac{5k^2}{4}&=0 \\
	-25k^2+80+5k^2&=0 \\
	20k^2&=80 \\
	k&=\pm2
\end{align*}
The lamp cannot be underneath the ground, so the only solution for k is 2. Hence the lamp is located \boxed{2} units above the $x$-axis.
\end{minipage}
\flushleft


\section*{Problem 8}
\begin{align*}
	x^my^n&=(x+y)^{m+n} \\
	\ln\left(x^my^n\right)&=\ln\left((x+y)^{m+n}\right) \\
	m\ln x+n\ln y&=(m+n)\ln(x+y) \\
	\diff{}{x}\left[m\ln x+n\ln y\right]&=\diff{}{x}\left[(m+n)\ln(x+y)\right] \\
	\frac{m}{x}+\frac{n}{y}\left(\diff{y}{x}\right)&=(m+n)\left(\frac{1}{x+y}\right)\left(1+\diff{y}{x}\right) \\
	\frac{m}{x}+\frac{n}{y}\left(\diff{y}{x}\right)&=\frac{m+n}{x+y}+\frac{m+n}{x+y}\left(\diff{y}{x}\right) \\
	\frac{m}{x}-\frac{m+n}{x+y}&=\left(\frac{m+n}{x+y}-\frac{n}{y}\right)\left(\diff{y}{x}\right) \\
	\frac{m(x+y)-x(m+n)}{x(x+y)}&=\left(\frac{y(m+n)-n(x+y)}{y(x+y)}\right)\left(\diff{y}{x}\right) \\
	\frac{\cancel{mx}+my-\cancel{mx}-nx}{x\cancel{(x+y)}}&=\left(\frac{my+\cancel{ny}-nx-\cancel{ny}}{y\cancel{(x+y)}}\right)\left(\diff{y}{x}\right) \\
	\frac{\cancel{my-nx}}{x}&=\left(\frac{\cancel{my-nx}}{y}\right)\left(\diff{y}{x}\right) \\
	\Aboxed{\diff{y}{x}&=\frac{y}{x}}
\end{align*}
\qed


\section*{Problem 9}
\begin{itemize}
\item[(a)]
\begin{align*}
	(fg)''&=\diff{}{x}\left[\diff{}{x}(fg)\right] \\
	&=\diff{}{x}\left[f'g+fg'\right] \\
	&=\diff{}{x}\left[f'g\right]+\diff{}{x}\left[fg'\right] \\
	&=f''g+f'g'+f'g'+fg'' \\
	\Aboxed{(fg)''&=f''g+2f'g'+fg''}
\end{align*}
\qed

\item[(b)]
\begin{proof}
	By induction.

	\textbf{Base case.} $n=1$
	\begin{gather*}
		(fg)'=f'g+fg' \\
		\sum_{k=0}^{1}\binom{1}{k}f^{(k)}g^{(n-k)}=\binom{1}{0}f^{(0)}g^{(1-0)}+\binom{1}{1}f^{(1)}g^{(1-0)}=fg'+f'g
	\end{gather*}

	\textbf{Induction Hypothesis.} Suppose that:
	\setlength{\abovedisplayskip}{0pt}
	\begin{equation*}
		(fg)^{(n)}=\sum_{k=0}^{n}\binom{n}{k}f^{(k)}g^{(n-k)}
	\end{equation*}
	We will show that:
	\begin{equation*}
		(fg)^{(n+1)}=\sum_{k=0}^{n+1}\binom{n+1}{k}f^{(k)}g^{(n-k+1)}
	\end{equation*}

\textbf{Inductive Step.}

\begin{align*}
	(fg)^{(n+1)}&=\diff{}{x}\left[\sum_{k=0}^{n}\binom{n}{k}f^{(k)}g^{(n-k)}\right] \\
	&=\sum_{k=0}^{n}\binom{n}{k}\diff{}{x}\left[f^{(k)}g^{(n-k)}\right] \\
	&=\sum_{k=0}^{n}\binom{n}{k}\left(f^{(k+1)}g^{(n-k)}+f^{(k)}g^{(n-k+1)}\right) \\
	&=\sum_{k=0}^{n}\binom{n}{k}f^{(k+1)}g^{(n-k)}+\sum_{k=0}^{n}\binom{n}{k}f^{(k)}g^{(n-k+1)} \\
	&=\sum_{k=1}^{n+1}\binom{n}{k-1}f^{(k)}g^{(n-k+1)}+\sum_{k=1}^{n+1}\binom{n}{k}f^{(k)}g^{(n-k+1)}+\binom{n}{0}f^{(0)}g^{(n+1)}-\binom{n}{n+1}f^{(n+1)}g^{(n-(n+1)+1)} \\
	&=\sum_{k=1}^{n+1}\left(\binom{n}{k-1}+\binom{n}{k}\right)f^{(k)}g^{(n-k+1)}+f^{(0)}g^{(n+1)} \\
	&=\sum_{k=1}^{n+1}\left(\frac{n!}{(n-(k-1))!(k-1)!}+\frac{n!}{k!(n-k)!}\right)f^{(k)}g^{(n-k+1)}+f^{(0)}g^{(n+1)} \\
	&=\sum_{k=1}^{n+1}\left(\frac{n!}{(n-k+1)(n-k)!(k-1)!}+\frac{n!}{k(k-1)!(n-k)!}\right)f^{(k)}g^{(n-k+1)}+f^{(0)}g^{(n+1)} \\
	&=\sum_{k=1}^{n+1}\left(\frac{(k)(n!)+(n-k+1)(n!)}{k(n-k+1)(n-k)!(k-1)!}\right)f^{(k)}g^{(n-k+1)}+f^{(0)}g^{(n+1)} \\
	&=\sum_{k=1}^{n+1}\left(\frac{(n-\cancel{k}+1+\cancel{k})(n!)}{((n-k+1)(n-k)!)(k(k-1)!)}\right)f^{(k)}g^{(n-k+1)}+\binom{n+1}{0}f^{(0)}g^{(n-0+1)} \\
	&=\sum_{k=1}^{n+1}\left(\frac{(n+1)!}{((n+1)-k)!(k!)}\right)f^{(k)}g^{(n-k+1)}+\binom{n+1}{0}f^{(0)}g^{(n-0+1)} \\
	&=\sum_{k=1}^{n+1}\binom{n+1}{k}f^{(k)}g^{(n-k+1)}+\binom{n+1}{0}f^{(0)}g^{(n-0+1)} \\
	&=\sum_{k=0}^{n+1}\binom{n+1}{k}f^{(k)}g^{(n-k+1)}
\end{align*}
\end{proof}

\end{itemize}

\end{document}