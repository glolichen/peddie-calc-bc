\documentclass{article}

\usepackage[letterpaper,portrait,top=0.4in, left=0.8in, right=0.8in, bottom=1in]{geometry}

\usepackage{amsmath, amsfonts, amsthm, amssymb}
\usepackage{graphicx, float}
\usepackage{mathtools}
\usepackage{siunitx}
\usepackage{titlesec}
\usepackage{interval}
\usepackage{titling}
\usepackage{multicol}
\usepackage{siunitx}
\usepackage{vwcol}
\usepackage{empheq}
\usepackage{cancel}

\intervalconfig {
	soft open fences
}

\newcommand{\alignedintertext}[1]{%
	\noalign{%
		\vskip\belowdisplayshortskip
		\vtop{\hsize=\linewidth#1\par
		\expandafter}%
		\expandafter\prevdepth\the\prevdepth
	}%
}

%opening
\title{Finish Problem Set \#40}
\author{Jayden Li}
\date{January 11, 2024}

\begin{document}

\newgeometry{top=0.4in, left=0.8in, right=0.8in, bottom=1in}

\fontsize{12pt}{12pt}\selectfont

\maketitle

\section*{Problem 4}
\centering
\begin{minipage}[t]{0.45\linewidth}
\setlength{\abovedisplayskip}{0pt}
\begin{align*}
	\sum_{n=0}^{\infty}\sum_{m=0}^{n}\frac{1}{2^{m+n}}&=
		\sum_{n=0}^{\infty}\left(\sum_{m=0}^{n}\frac{1}{2^m}
		\frac{1}{2^n}\right) \\
	&=\sum_{n=0}^{\infty}\frac{1}{2^n}\sum_{m=0}^{n}\frac{1}{2^m} \\
	&=\sum_{n=0}^{\infty}\frac{1}{2^n}\left(\frac{1}{2^0}\left(
		\frac{1-\left(\frac{1}{2}\right)^{n+1}}{1-\frac{1}{2}}
		\right)\right) \\
	&=\sum_{n=0}^{\infty}\frac{2}{2^n}\left(
		1-\frac{1}{2^{n+1}}\right) \\
	&=\sum_{n=0}^{\infty}\left(\frac{2}{2^n}-\frac{2}{2^n2^{n+1}}
		\right) \\
	&=2\sum_{n=0}^{\infty}\left(\frac{1}{2}\right)^n-
		2\sum_{n=0}^{\infty}\frac{1}{2^{2n+1}}
\end{align*}
\end{minipage}
\begin{minipage}[t]{0.45\linewidth}
\setlength{\abovedisplayskip}{0pt}
\begin{align*}
	&=2\left(\frac{1}{1-\frac{1}{2}}\right)-2\sum_{n=0}^{\infty}
		\left(\left(\frac{1}{2}\right)^2\right)^{n+\frac{1}{2}} \\
	&=4-2\sum_{n=0}^{\infty}\left(\frac{1}{4}\right)^{n+\frac{1}{2}} \\
	&=4-2\left(\frac{\left(\frac{1}{4}\right)^{\frac{1}{2}}}
		{1-\frac{1}{4}}\right) \\
	&=4-2\left(\frac{1}{2}\cdot\frac{4}{3}\right) \\
	&=\frac{12}{3}-\frac{4}{3} \\
	&=\boxed{\frac{8}{3}}
\end{align*}
\end{minipage}
\flushleft

\section*{Problem 5}
\begin{minipage}[t]{0.45\linewidth}
\setlength{\abovedisplayskip}{0pt}
\begin{align*}
	4\sum_{k=2}^{\infty}(1+n)^{-k}&=27n
	\intertext{
		\setlength{\abovedisplayskip}{0pt}\begin{align*}
		&\implies-1<1+n<1 \\
		&\implies-2<n<0
	\end{align*}}
	4\sum_{k=2}^{\infty}\left(\frac{1}{1+n}\right)^{k}&=27n \\
	4\left(\frac{\left(\frac{1}{1+n}\right)^{2}}{1-\frac{1}{1+n}}
		\right)&=27n
\end{align*}
\end{minipage}
\begin{minipage}[t]{0.45\linewidth}
\setlength{\abovedisplayskip}{0pt}
\begin{align*}
	4\left(\frac{\frac{1}{(1+n)^2}}{\frac{(1+n)^2}{(1+n)^2}-
		\frac{1+n}{(1+n)^2}}\right)&=27n \\
	4\left(\frac{1}{(1+n)^2-1-n}\right)&=27n \\
	\frac{4}{1+2n+n^2-1-n}&=27n \\
	\frac{4}{n^2+n}&=27n \\
	27n^3+27n^2-4&=0 \\
\end{align*}
\end{minipage}
\pagebreak

Rational roots: $\pm1,2,4,\dfrac{1}{3},\dfrac{2}{3},\dfrac{4}{3},
\dfrac{1}{9},\dfrac{2}{9},\dfrac{4}{9},\dfrac{1}{27},\dfrac{2}{27},
\dfrac{4}{27}$
\begin{align*}
	27\left(\frac{1}{3}\right)^3+27\left(\frac{1}{3}\right)^2-4&=0 \\
	27\left(-\frac{2}{3}\right)^3+27\left(-\frac{2}{3}\right)^2-4&=0
\end{align*}
The restriction on $n$ so that the geometric series converges is
$-2<n<0$. $\dfrac{1}{3}\geq0$, so the only solution is
$n=-\dfrac{2}{3}$.
\qed

\title{Progress on Problem Set \#41}
\maketitle

\section*{Problem 1}
\begin{align*}
	\frac{12x+11}{(x-2)(2x+3)}&=\frac{A}{x-2}+\frac{B}{2x+3} \\
	\frac{12x+11}{(x-2)(2x+3)}&=\frac{A(2x+3)+B(x-2)}{(x-2)(2x+3)} \\
	\frac{12x+11}{(x-2)(2x+3)}&=\frac{2Ax+Bx+3A-2B}{(x-2)(2x+3)} \\
	12x+11&=(2A+B)x+(3A-2B) \\
\end{align*}
\centering
\begin{minipage}{0.3\linewidth}
\setlength{\abovedisplayskip}{0pt}
\begin{empheq}[left=\empheqlbrace]{align*}
	\displaystyle &12=2A+B \\
	\displaystyle &11=3A-2B
\end{empheq}
\begin{empheq}[left=\empheqlbrace]{align}
	\displaystyle &24=4A+2B \\
	\displaystyle &11=3A-2B
\end{empheq}
\end{minipage}
\begin{minipage}{0.4\linewidth}
\setlength{\abovedisplayskip}{0pt}
\begin{align*}
	24+11&=4A+2B+3A-2B \\
	7A&=35 \\
	A&=5
\end{align*}
\begin{align*}
	11&=3(5)-2B \\
	2B&=4 \\
	B&=2
\end{align*}
\end{minipage}
\flushleft
\begin{equation*}
	\boxed{A=5,B=2}
\end{equation*}

\section*{Problem 2}
\begin{itemize}
\centering
\begin{minipage}[t]{0.55\linewidth}
\setlength{\abovedisplayskip}{0pt}
\item[(a)]
$\begin{aligned}[t]
	&\left(1-\frac{1}{3}\right)\left(1-\frac{1}{4}\right)\left(
		1-\frac{1}{5}\right)\cdots\left(1-\frac{1}{n}\right) \\
	=\,&\left(\frac{2}{\cancel{3}}\right)
		\left(\frac{\cancel{3}}{\cancel{4}}\right)
		\left(\frac{\cancel{4}}{\cancel{5}}\right)
		\cdots\left(\frac{\cancel{n-1}}{n}\right) \\
	=\,&\boxed{\frac{2}{n}}
\end{aligned}$
\end{minipage}
\begin{minipage}[t]{0.4\linewidth}
\setlength{\abovedisplayskip}{0pt}
\item[(b)]
$\begin{aligned}[t]
	&\sum_{n=1}^{99}\left(\frac{1}{n}-\frac{1}{n+1}\right) \\
	=&\,\frac{1}{1}-\cancel{\frac{1}{2}+\frac{1}{2}}\cancel{
		-\frac{1}{3}+\cdots+\frac{1}{99}}-\frac{1}{100} \\
	=&\,\frac{100}{100}-\frac{1}{100} \\
	=&\,\boxed{\frac{99}{100}}
\end{aligned}$
\end{minipage}
\flushleft

\item[(c)]
\centering
\begin{minipage}[t]{0.5\linewidth}
\setlength{\abovedisplayskip}{0pt}
$\begin{aligned}[t]
	&\frac{1}{1\cdot2}+\frac{1}{2\cdot3}+\cdots+\frac{1}{99\cdot100} \\
	=&\,\sum_{n=1}^{99}\frac{1}{n(n+1)} \\
	=&\,\sum_{n=1}^{99}\left(\frac{A}{n}-\frac{B}{n+1}\right) \\
\end{aligned}$
\begin{align*}
	\intertext{	\setlength{\abovedisplayskip}{0pt}\begin{align*}
		A(n+1)-Bn&=1 \\
		\intertext{There are infinitely many solutions for $A$ and
		$B$. To ensure telescoping works, add another condition:}
		A&=B \\
		An+A-An&=1 \\
		A,B&=1
	\end{align*}}
\end{align*}
\end{minipage}
\begin{minipage}[t]{0.4\linewidth}
\setlength{\abovedisplayskip}{0pt}
\begin{align*}
	\qquad
	=&\,\sum_{n=1}^{99}\left(\frac{1}{n}-\frac{1}{n+1}\right) \\
	=&\,\frac{1}{1}-\frac{1}{2}+\frac{1}{2}-\frac{1}{3}+\cdots+
		\frac{1}{99}-\frac{1}{100} \\
	=&\,\frac{100}{100}-\frac{1}{100} \\
	=&\,\boxed{\frac{99}{100}}
\end{align*}
\end{minipage}
\flushleft

\end{itemize}

\end{document}
