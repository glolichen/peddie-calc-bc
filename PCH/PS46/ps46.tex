\documentclass{article}

\usepackage[letterpaper,portrait,top=0.4in, left=0.6in, right=0.6in, bottom=1in]{geometry}

\usepackage{amsmath, amsfonts, amsthm, amssymb}
\usepackage{graphicx, float}
\usepackage{mathtools}
\usepackage{titlesec}
\usepackage{interval}
\usepackage{hyperref}
\usepackage{titling}
\usepackage{vwcol}
\usepackage{setspace}
\usepackage{empheq}
\usepackage{cancel}
\usepackage{esdiff}
\usepackage{multicol}
\usepackage{mdframed}

\intervalconfig {
	soft open fences
}

\newcommand{\alignedintertext}[1]{%
	\noalign{%
		\vskip\belowdisplayshortskip
		\vtop{\hsize=\linewidth#1\par
		\expandafter}%
		\expandafter\prevdepth\the\prevdepth
	}%
}

% \allowdisplaybreaks

%opening
\title{Problem Set \#46}
\author{Jayden Li}
\date{January 31, 2024}

\begin{document}

\fontsize{12pt}{12pt}\selectfont

\maketitle

\section*{Problem 1}
\begin{itemize}
\item[(a)]
    \setlength{\abovedisplayskip}{0pt}
	\begin{align*}
        PF_1-PF_2&=\pm2a \\
		\sqrt{(x-(-c))^2+(y-0)^2}-\sqrt{(x-c)^2+(y-0)^2}&=\pm2a \\
		\left(\sqrt{x^2+2xc+c^2+y^2}-\sqrt{x^2-2xc+c^2+y^2}\right)^2&=(\pm2a)^2 \\
		\begin{aligned}
			x^2+2xc+c^2+y^2+x^2-2xc+c^2+y^2& \\
			-2\sqrt{\left(x^2+2xc+c^2+y^2\right)\left(x^2-2xc+c^2+y^2\right)}&
		\end{aligned}&=4a^2 \\
		2\sqrt{\left(x^2+2xc+c^2+y^2\right)\left(x^2-2xc+c^2+y^2\right)}&=2x^2+2c^2+2y^2-4a^2 \\
		\left(x^2+2xc+c^2+y^2\right)\left(x^2-2xc+c^2+y^2\right)&=\left(x^2+c^2+y^2-2a^2\right)^2 \\
		\begin{aligned}
			\bcancel{x^4}-\cancel{2x^3c}+\bcancel{x^2c^2}+\bcancel{x^2y^2}+& \\
			\cancel{2x^3c}-4x^2c^2+\cancel{2xc^3}+\cancel{2xcy^2}+& \\
			\bcancel{x^2c^2}-\cancel{2xc^3}+\bcancel{c^4}+\bcancel{y^2c^2}+& \\
			\bcancel{x^2y^2}-\cancel{2xcy^2}+\bcancel{y^2c^2}+\bcancel{y^4}&
		\end{aligned}&=
		\begin{aligned}
			&\bcancel{x^4}+\bcancel{x^2c^2}+\bcancel{x^2y^2}-2x^2a^2+ \\
			&\bcancel{x^2c^2}+\bcancel{c^4}+\bcancel{y^2c^2}-2c^2a^2+ \\
			&\bcancel{x^2y^2}+\bcancel{y^2c^2}+\bcancel{y^4}-2y^2a^2- \\
			&2x^2a^2-2c^2a^2-2y^2a^2+4a^4
		\end{aligned} \\
		2x^2a^2+2c^2a^2+2y^2a^2+2x^2a^2+2c^2a^2+2y^2a^2-4a^4&=4x^2c^2 \\
		2x^2a^2+2\left(a^2+b^2\right)a^2+2y^2a^2-2a^4&=2x^2\left(a^2+b^2\right) \\
		\cancel{x^2a^2}+\cancel{a^4}+a^2b^2+y^2a^2-\cancel{a^4}&=\cancel{x^2a^2}+x^2b^2 \\
		x^2b^2-y^2a^2&=a^2b^2 \\
		\Aboxed{\frac{x^2}{a^2}-\frac{y^2}{b^2}&=1} \tag{1}
    \end{align*}

\item[(b)]
The hyperbola given by equation (1) is centered at the origin $(0,0)$. A hyperbola centered at a point $(h,k)$ is given by applying to (1) a horizontal translation of $h$ units and a vertical translation of $k$ units. Let $(x_1,y_1)$ be members of the solution set of the new hyperbola. Because the new hyperbola is given by a translation to (1), we have $x_1=x+h,y_1=y+k\implies x=x_1-h,y=y_1-k$. Substituting this to (1) gives:
\begin{equation*}
	\frac{(x_1-h)^2}{a^2}-\frac{(y_1-k)^2}{b^2}=1
\end{equation*}
Thus, the equation of the new hyperbola is:
\begin{equation*}
	\boxed{\frac{(x-h)^2}{a^2}-\frac{(y-k)^2}{b^2}=1} \tag{2}
\end{equation*}

\item[(c)]
First, we will find the equation of a hyperbola centered at the origin whose transverse axis is vertical. This hyperbola is a reflection of the hyperbola given by (1) across the line $y=x$. This is equivalent to the inverse of (1), which is given by switching the $x$ and $y$ variables.
\begin{equation*}
	\frac{y^2}{a^2}-\frac{x^2}{b^2}=1
\end{equation*}
To find the equation centered at $(h,k)$ we apply a horizontal translation of $h$ units and a vertical translation of $k$ units.
\begin{equation*}
	\boxed{\frac{(y-k)^2}{a^2}-\frac{(x-h)^2}{b^2}=1} \tag{3}
\end{equation*}

\end{itemize}

\section*{Problem 2}
\begin{itemize}
\item[(a)]
	\setlength{\abovedisplayskip}{0pt}
	\begin{minipage}[t]{0.43\linewidth}
		\begin{align*}
			\frac{x^2}{a^2}-\frac{y^2}{b^2}&=1 \\
			x^2b^2-y^2a^2&=a^2b^2 \\
			y^2a^2&=x^2b^2-a^2b^2 \\
			y^2&=\frac{x^2b^2}{a^2}-b^2 \\
			y&=\pm\sqrt{\frac{b^2}{a^2}\left(x^2-a^2\right)} \\
			\Aboxed{y&=\pm\frac{b}{a}\sqrt{x^2-a^2}}
		\end{align*}
	\end{minipage}
	\begin{minipage}[t]{0.56\linewidth}
		\begin{mdframed}
			Explicit formula for $y$ from (2):
			\begin{align*}
				\frac{(x-h)^2}{a^2}-\frac{(y-k)^2}{b^2}&=1 \\
				(x-h)^2b^2-(y-k)^2a^2&=a^2b^2 \\
				(y-k)^2a^2&=(x-h)^2b^2-a^2b^2 \\
				(y-k)^2&=\frac{(x-h)^2b^2}{a^2}-b^2 \\
				y-k&=\pm\sqrt{\frac{b^2}{a^2}\left((x-h)^2-a^2\right)} \\
				\Aboxed{y&=\pm\frac{b}{a}\sqrt{(x-h)^2-a^2}+k}
			\end{align*}
		\end{mdframed}
	\end{minipage}

	\qed

\item[(b)]
\begin{itemize}
	\begin{multicols}{2}
		\item[i.]
		\begin{align*}
			y&\to\lim_{x\to\infty}\pm\frac{b}{a}\sqrt{x^2-a^2} \\
			y&\to\pm\lim_{x\to\infty}\frac{b}{a}\sqrt{x^2\left(1-\cancel{\frac{a^2}{x^2}}\right)} \\
			y&\to\pm\lim_{x\to\infty}\frac{b|x|}{a} \\
			\Aboxed{y&\to\pm\infty}
		\end{align*}
		\item[ii.]
		\begin{align*}
			y&\to\lim_{x\to\infty}\left(\pm\frac{b}{a}\sqrt{(x-h)^2-a^2}+k\right) \\
			y&\to\pm\lim_{x\to\infty}\frac{b}{a}\sqrt{(x-h)^2\left(1-\cancel{\frac{a^2}{(x-h)^2}}\right)}+k \\
			y&\to\pm\lim_{x\to\infty}\frac{b|x-h|}{a}+k \\
			\Aboxed{y&\to\pm\infty}
		\end{align*}
	\end{multicols}
\end{itemize}

\item[(c)]
\begin{minipage}[t]{0.51\linewidth}
	\begin{align*}
		\frac{(y-k)^2}{a^2}-\frac{(x-h)^2}{b^2}&=1 \\
		(y-k)^2b^2-(x-h)^2a^2&=a^2b^2 \\
		(y-k)^2&=\frac{a^2b^2+(x-h)^2a^2}{b^2} \\
		y-k&=\pm\sqrt{\frac{a^2}{b^2}\left(b^2+(x-h)^2\right)} \\
		y&=\pm\frac{a}{b}\sqrt{b^2+(x-h)^2}+k \\
	\end{align*}
\end{minipage}
\begin{minipage}[t]{0.48\linewidth}
	\begin{align*}
		y&\to\lim_{x\to\infty}\left(\pm\frac{a}{b}\sqrt{b^2+(x-h)^2}+k\right) \\
		y&\to\pm\lim_{x\to\infty}\frac{a}{b}\sqrt{(x-h)^2\left(\cancel{\frac{b^2}{(x-h)^2}}+1\right)}+k \\
		y&\to\pm\lim_{x\to\infty}\frac{a|x-h|}{b}+k \\
		\Aboxed{y&\to\pm\infty}
	\end{align*}
\end{minipage}
The behavior of $y$ as $x\to\infty$ does not change.

\item[(d)]
	\begin{align*}
		e&=\frac{c}{a} \\
		e^2&=\frac{a^2+b^2}{a^2} \\
		e^2&=1+\frac{b^2}{a^2} \\
		e&=\pm\sqrt{1+\frac{b^2}{a^2}}
		\intertext{Eccentricity must be positive.}
		\Aboxed{e&=\sqrt{1+\frac{b^2}{a^2}}}
	\end{align*}

% \begin{multicols*}{2}
% 	\item[(d)]
% 		\begin{align*}
% 			e&=\frac{c}{a} \\
% 			e^2&=\frac{a^2+b^2}{a^2} \\
% 			e^2&=1+\frac{b^2}{a^2} \\
% 			e&=\pm\sqrt{1+\frac{b^2}{a^2}}
% 			\intertext{Eccentricity must be positive.}
% 			\Aboxed{e&=\sqrt{1+\frac{b^2}{a^2}}}
% 		\end{align*}
% 	\vfill\null\columnbreak
% 	\item[(e)]
% 		$\begin{aligned}[t]
% 			4x^2-9y^2&=36 \\
% 			\frac{x^2}{9}-\frac{y^2}{4}&=1
% 			\implies a=3,b=2,c=\sqrt{13},\text{horizontal}
% 		\end{aligned}$

% 		Center $(0,0)$, Foci $(\sqrt{13},0),(-\sqrt{13},0)$,\\ Vertices $(3,0),(-3,0)$, Asymptotes $y=\pm\dfrac{2}{3}x$,\\ Eccentricity $\dfrac{\sqrt{13}}{3}$
% \end{multicols*}

% $\begin{aligned}[t]
% 	x^2+12x-3y^2+18y+21&=0 \\
% 	(x^2+12x)-3(y^2-6y)+21&=0 \\
% 	((x+6)^2-36)-3((y-3)^2-9)+21&=0 \\
% 	(x+6)^2-36-3(y-3)^2+27+21&=0 \\
% 	(x+6)^2-3(y-3)^2&=-12 \\
% 	\frac{(y-3)^2}{4}-\frac{(x+6)^2}{12}&=1 \\
% 	\implies a=3,b=2,c=\sqrt{13},\text{horizontal}
% \end{aligned}$

% Center $(0,0)$, Foci $(\sqrt{13},0),(-\sqrt{13},0)$,\\ Vertices $(3,0),(-3,0)$, Asymptotes $y=\pm\dfrac{2}{3}x$,\\ Eccentricity $\dfrac{\sqrt{13}}{3}$
\end{itemize}

\end{document}